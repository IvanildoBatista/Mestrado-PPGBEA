\documentclass[a4paper,notitlepage]{book}
\usepackage[lmargin=2cm,tmargin=2cm,rmargin=2cm,bmargin=2cm]{geometry}
\usepackage[utf8]{inputenc}
\usepackage{amsmath}
\usepackage{geometry}
\usepackage[brazilian]{babel}
\usepackage{xcolor}
\usepackage{setspace}
\usepackage[inline]{enumitem}
\usepackage{ulem}
\usepackage{pifont}
%\usepackage{xcolor}
%\usepackage{setspace}
%\usepackage{enumitem}
%usepackage{ulem}
%\usepackage{pifont}

\title{Atividade 2}
\author{Ivanildo Batista e Mackis Lima}
\date{\today}

\begin{document}

\maketitle

\begin{section}{Exercícios Propostos}

\begin{subsection}{De uma população normal X com variância 121 retiramos uma amostra de 25 observações, obtendo $\overline{x}$ = 45. Ao nível de 2\% fazer um IC para a verdadeira média da população X.}

\begin{align*}
 \mathbf{\sigma_{\overline{x}} = \dfrac{11}{5} = 2,2 \qquad z_{\alpha} = z_{1\%} = 2,33} \\
 \mathbf{P(45 - 2,33.2,2 < \mu < 45+2,33.2,2) = 0,98}\\
 \mathbf{P(45-5,126 < \mu < 45+5,126) = 0,98}\\
 \mathbf{P(39,874 < \mu < 50,126) = 0,98}
\end{align*}


\end{subsection}

\begin{subsection}{Levanta-se uma amostra de 10 observações de uma população normal com variância 160, obtendo-se $\sum_{i=1}^{10} x_i = 2300$. Determinar os IC para a média $\mu$ aos níveis de 20\% e 10\%.} 
\end{subsection}

\end{section}