\documentclass[a4paper,notitlepage]{article}
\usepackage[lmargin=2cm,tmargin=2cm,rmargin=2cm,bmargin=2cm]{geometry}
\usepackage[utf8]{inputenc}
\usepackage{geometry}
\usepackage[brazilian]{babel}
\usepackage{xcolor}
\usepackage{setspace} 
\usepackage[inline]{enumitem}
\usepackage{ulem}
\usepackage{pifont}

\begin{document}

\title{Uso de Editores de Textos na Elaboração de Documentos Científicos}
\author{Ivanildo Batista da Silva Júnior}
\date{13 de setembro de 2021}

\maketitle

\doublespacing

\begin{center}
\begin{minipage}{0.75\linewidth}
\textbf{\textcolor{blue}{Instruções gerais}} para elaboração deste documento estão apresentadas aqui: foi utilizado um documento do tipo artigo com \underline{uma coluna}, papel a4, margem de 2cm (esquerda, direita, superior e inferior); comandos para inserção de \textit{título}, \textbf{autor} e \textsc{data} foram utilizados, \textcolor{red}{faça pesquisa de quais são esses comandos e utilize-os!!!}; o \underline{espaçamento} entre linhas é duplo. Este texto está em uma caixa de texto centralizada com largura igual a 75\% da largura do texto. O nome do discente deve ser substituído em “Seu nome”. \textcolor{red}{\ding{110}}
\end{minipage}
\end{center}

Para garantir a formatação os seguintes \colorbox{gray!80}{pacotes} foram utilizados:

\begin{enumerate*}[label =(\roman*)., itemjoin =\hfill]
\item \textcolor{blue}{babel} 
\item \textcolor{blue}{geometry} 
\item \textcolor{blue}{xcolor} 
\item \textcolor{blue}{setspace} 
\item \textcolor{blue}{enumitem} 
\item \textcolor{blue}{ulem} 
\item \textcolor{blue}{pifont}
\end{enumerate*}

%
\fcolorbox{red}{gray}{\textcolor{green}{
\begin{minipage}{0.95\linewidth} 
!!!Nenhum pacote — além dos descritos acima — devem ser utilizados na elaboração deste documento!!!
\end{minipage}
}}

\vspace{1cm}

\fbox{\begin{minipage}[b]{4.5cm}
Este parágrafo está contido em uma caixa com 4.5 centímetros de largura. Você acreditaria que o texto que
está acima e abaixo desta caixa dista um centímetro da caixa?
\end{minipage}} \qquad \fbox{\begin{minipage}[b]{4.5cm}
Este parágrafo está contido em uma caixa com 4.5 centímetros de largura. Você acreditaria que o texto que
está acima e abaixo desta caixa dista um centímetro da caixa?
\end{minipage}} \qquad \fbox{\begin{minipage}[b]{4.5cm}
Este parágrafo está contido em uma caixa com 4.5 centímetros de largura. Você acreditaria que o texto que
está acima e abaixo desta caixa dista um centímetro da caixa?
\end{minipage}}

\vspace{1cm}

A forma como o \textcolor{blue}{\LaTeX} trata o espaçamento entre \sout{palavras} é realmente curioso! \hspace{2cm} Este espaço em branco mede dois centímetros.

\vspace{0.5cm}
\begin{center}
\fcolorbox{red}{green!10}{\begin{minipage}[b]{7.62cm}
Este parágrafo está contido em uma caixa com três polegadas de largura. 10\% de verde foi utilizado para preencher o fundo e a cor vermelha para contornar a caixa. A caixa está centralizada e a distância entre ela e o texto acima é de 0.5 centímetro.
\end{minipage}}
\end{center} 

\makebox[1\textwidth][s]{O fim do documento.} 

\textcolor{red}{\ding{168}} \textcolor{blue}{\hrulefill} \textcolor{red}{\ding{168}}
\end{document}
