\documentclass[12pt,a4paper,brazil]{article}
\usepackage[utf8]{inputenc}
\usepackage[brazil]{babel}
\usepackage{hyperref}
\usepackage{float}
\usepackage{graphicx}
\usepackage[alf,bibjustif]{abntex2cite}
\usepackage{lipsum}
\usepackage{xcolor}
\usepackage{soul}
\usepackage{todonotes}
\usepackage{amsmath,amssymb}
\usepackage{pgfplots}
\usepackage{pgfplotstable}
\usepackage{indentfirst}
\usepackage{cancel}
\usepackage{xcolor}
\usepackage{listings}
\usepackage{subfig}
\usepackage{multicol}
\usepackage{easyReview}
\renewcommand\lstlistingname{Código}
\renewcommand\lstlistlistingname{Código}


%\renewcommand{\alertColor}{\textcolor{blue}}
%\renewcommand{\removeColor}{\textcolor{orange}}
%\renewcommand{\addColor}{\textcolor{pink}}



%\lstset{language=TeX,
 %   basicstyle=\small\ttfamily,
  %  stringstyle=\color{blue}%!70!black},
   % otherkeywords={0,1,2,3,4,5,6,7,8,9},
%    morekeywords={TRUE,FALSE},
 %   deletekeywords={data,frame,length,as,character},
  %  backgroundcolor =\color{blue!5!white} ,
  %  keywordstyle=\color{blue},
  %  commentstyle=\color{green!7!black},
%}

\lstset{language=tex,
		keywordstyle=\bfseries\color{purple!60!black},
		commentstyle=\itshape\color{green!40!black},
  		%identifierstyle=\itshape,
		stringstyle=\itshape,
		showstringspaces=false,
		escapeinside={\%*}{*)},
		columns=flexible,
		breaklines=true,
		breakindent=0pt}
\newcommand{\keyword}[1]{\textbf{\color{purple!60!black}{#1}}}
\hypersetup{
	colorlinks=true,
	linkcolor=blue,
	citecolor=blue,
	urlcolor=blue
}

\begin{document}

\noindent O bonde passa cheio de pernas:\\
pernas brancas pretas amarelas.\\
Para que tanta perna, meu Deus,\\
pergunta meu coração.\\
Porém meus olhos\\
não perguntam nada

\setreviewson

Quando nasci, um anjo torto
desses que vivem na sombra
disse: Vai, Carlos! ser gauche na vida.


\noindent \comment{Quando nasci, um anjo torto
desses que vivem na sombra
disse: Vai, Carlos! ser gauche na vida.}{Poema de Carlos Drummond de Andrade}

Você pode ter defeitos, viver \alert{anssioso} e ficar irritado, algumas vezes, mas não se esqueça de que sua vida é a maior \alert{empreza} do mundo.

Só você pode evitar que ela vá à \alert{fal{\^e}nsia}. Há muitas pessoas que precisam, admiram e \alert{tor{\c c}em} por você.

Gostaria que você sempre se \alert{lembra{\c c}e} de que ser feliz não é ter um céu sem tempestades, caminhos sem \alert{assidentes}, trabalhos sem fadigas, relacionamentos sem \alert{decepss{\~o}es}. Ser feliz é encontrar força no perdão, esperança nas batalhas, segurança no palco do medo, amor nos desencontros. Ser feliz não é apenas valorizar o sorriso, mas refletir sobre a tristeza.



Você pode ter defeitos, viver anssioso e ficar irritado, algumas vezes, mas não se esqueça de que sua vida é a maior empreza do mundo.

Só você pode evitar que ela vá à falênsia. Há muitas pessoas que precisam, admiram e torçem por você.

Gostaria que você sempre se lembraçe de que ser feliz não é ter um céu sem tempestades, caminhos sem assidentes, trabalhos sem fadigas, relacionamentos sem decepssões Ser feliz é encontrar força no perdão, esperança nas batalhas, segurança no palco do medo, amor nos desencontros. Ser feliz não é apenas valorizar o sorriso, mas refletir sobre a tristeza.\\

%%%%%%%%%%%%%%%%%%%%%%%%%%%%%%%%%%%%%%%%%%%%%%%%%%%%%%%%%%%%%%%%%%%%%%%%%%%%%%%%%%%%%%%%%%%%%%%%%%%%%%%%%%%%%%%%%%%%%%%%%%%%%%%%%%%%%%%%%%%%%%%%%%%%%%%%%%%%%%%%%%%%%%%%%%%%%%%%%%%%%%%%%%%%%%%%%%%%%%%%%%%%%%%%%%%%%%%%%%%%%%%%%%%%%%%%%%%%%%%%%%%%%%%%%%%%%%%%%%%%%%%%%%%%%%%%%%%%%%%%%%%%%%%%%%%%%%%%%%%%%%%%%%%%%%%%%%%%

A macroeconomia encara as coisas de uma forma mais ampla, olha para o grande cenário. Se você estudar macroeconomia, perceberá que as maiores preocupações dessa área estão relacionadas aos Estados, às economias nacionais e à relações econômicas internacionais. É a partir dessa análise macroeconômica que surgem indicadores muito conhecidos, que você provavelmente já deve ter ouvido falar: PIB (Produto Interno Bruto), inflação, juros, câmbio, balança comercial, entre tantos outros. Esses números são desenvolvidos a partir de análises amplas, que envolvem a produção econômica de um país inteiro, suas trocas com outros países e assim por diante.\\


\highlight{A macroeconomia encara as coisas de uma forma mais ampla, olha para o grande cen{\'a}rio}. Se você estudar macroeconomia, perceberá que as maiores preocupações dessa área estão relacionadas aos Estados, às economias nacionais e à relações econômicas internacionais. É a partir dessa análise macroeconômica que surgem indicadores muito conhecidos, que você provavelmente já deve ter ouvido falar: \highlight{PIB (Produto Interno Bruto), infla{\c c}{\~a}o, juros, {c\^a}mbio, balan{\c c}a comercial, entre tantos outros}. Esses números são desenvolvidos a partir de análises amplas, que envolvem a produção econômica de um país inteiro, suas trocas com outros países e assim por diante.

\newpage

%%%%%%%%%%%%%%%%%%%%%%%%%%%%%%%%%%%%%%%%%%%%%%%%%%%%%%%%%%%%%%%%%%%%%%%%%%%%%%%%%%%%%%%%%%%%%%%%%%%%%%%%%%%%%%%%%%%%%%%%%%%%%%%%%%%%%%%%%%%%%%%%%%%%%%%%%%%%%%%%%%%%%%%%%%%%%%%%%%%%%%%%%%%%%%%%%%%%%%%%%%%%%%%%%%%%%%%%

As energias renováveis são aquelas que dependem de processos em escala de tempo geológica para se tornarem disponíveis. Isso significa que, caso sejam esgotadas, demorarão muito tempo para se formarem novamente. Petróleo, carvão mineral e gás natural são os principais exemplos de fontes de energia renováveis.

As energias renováveis são \remove{aquelas que dependem de processos em escala de tempo geologica para se tornarem disponíveis. Isso significa que, caso sejam esgotadas, demorar{\~a}o muito tempo para se formarem novamente. Petroleo, carv{\~a}o mineral e gas natural s{\~a}o os principais exemplos de fontes de energias renovaveis}.\\

%%%%%%%%%%%%%%%%%%%%%%%%%%%%%%%%%%%%%%%%%%%%%%%%%%%%%%%%%%%%%%%%%%%%%%%%%%%%%%%%%%%%%%%%%%%%%%%%%%%%%%%%%%%%%%%%%%%%%%%%%%%%%%%%%%%%%%%%%%%%%%%%%%%%%%

Xadrez é um esporte. Pode ser classificado como um jogo de tabuleiro de natureza recreativa ou competitiva para dois jogadores.\\

Xadrez é um esporte \add{também considerado uma arte e uma ciência}. Pode ser classificado como um jogo de tabuleiro de natureza recreativa ou competitiva para dois jogadores \add{sendo também conhecido como Xadrez Ocidental ou Xadrez Internacional para distingui-lo dos seus antecessores e de outras variantes atuais}.

%%%%%%%%%%%%%%%%%%%%%%%%%%%%%%%%%%%%%%%%%%%%%%%%%%%%%%%%%%%%%%%%%%%%%%%%%%%%%%%%%%%%%%%%%%%%%%%%%%%%%%%%%%%%%%%%%%%%%%%%%%%%%%%%%%%%%%%%%%%%%%%%%%%%%%%%%%%%%%%%%%%%%%%%%%%%%%%%%%%%%%%%%%%%%%%%%%%%%%%%%%%%%%%%%%%%%%%%%%%%


Não é apenas comemorar as \substitute{conkistas}{conquistas}, mas aprender lições nos fracassos.

Não é apenas ter júbilo nos a \substitute{aplauzos}{aplausos}, mas encontrar alegria no anonimato.

Ser feliz é reconhecer que vale a pena viver a vida, \substitute{apezar}{apesar} de todos os desafios, \substitute{dezapontamentus}{desapontamentos} e períodos de crise. Ser feliz não é uma fatalidade do destino, mas uma conquista de quem sabe \substitute{viagar} para dentro do seu próprio ser.\\

\newpage

%%%%%%%%%%%%%%%%%%%%%%%%%%%%%%%%%%%%%%%%%%%%%%%%%%%%%%%%%%%%%%%%%%%%%%%%%%%%%%%%%%%%%%%%%%%%%%%%%%%%%%%%%%%%%%%%%%%%%%%%%%%%%%%%%%%%%%%%%%%%%%%%%%%%%%%%%%%%%%%%%%%%%%%%%%%%%%%%%%%%%%%%%%%%%%%%%%%%%%%%%%%%%%%%%%%%%%%%%%%%

Quando nasci, um anjo torto
desses que vivem na sombra
disse: Vai, Carlos! ser gauche na vida.


\comment{Quando nasci, um anjo torto
desses que vivem na sombra
disse: Vai, Carlos! ser gauche na vida.}{Poema de Carlos Drummond de Andrade}\\





\newpage 
\noindent O bonde passa cheio de pernas:\\
\highlight{pernas brancas pretas amarelas}.\\
\replace{Para que tanta perna, meu Deus,}{Qual o motivo de tanta gente, meu Deus ?}\\
pergunta meu coração.\\
\comment{Por{\'e}m meus olhos}{Pesquisar quem escreveu esse poema}
\noindent \remove{n{\~a}o perguntam nada}.




\newpage

\noindent Um texto sem o comando de alerta.\\
\alert{Um texto com o comando de alerta}.

Um texto sem o comando de destaque.
\highlight{Um texto com o comando de destaque}.\\

\noindent Um texto sem o comando de remoção.\\
\remove{Um texto com o comando de remo{\c c}{\~a}o}.\\

\noindent Este fragmento já estava no texto. 
\add{Essa parte foi adicionada agora no texto}.\\

\noindent \replace{Esta parte do texto precisa ser
substitu{\'i}do} {para esta parte mais recente do texto}\\

\noindent \substitute{Esta parte do texto precisa ser
substitu{\'i}do} {para esta parte mais recente do texto}\\


\setreviewson
\comment{Este texto receber{\'a} um
coment{\'a}rio.} {Este é o coment{\'a}rio que eu tenho!}


\setreviewsoff
\comment{Este texto receber{\'a} um
coment{\'a}rio.} {Este é o coment{\'a}rio que eu tenho!}


Você pode ter defeitos, viver \alert{anssioso} e ficar irritado, algumas vezes, mas não se esqueça de que sua vida é a maior \alert{empreza} do mundo.



\end{document}