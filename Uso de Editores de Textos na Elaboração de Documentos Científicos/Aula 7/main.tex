\documentclass[12pt,a4paper,brazil, openany, sumario=abnt-6027-2012]{abntex2}
\usepackage[utf8]{inputenc}
\usepackage[brazil]{babel}
\usepackage{hyperref}
\usepackage{float}
\usepackage{graphicx}
\usepackage[alf,bibjustif]{abntex2cite}
\usepackage{lipsum}
\usepackage{amsmath,amssymb}
\usepackage{pgfplots}
\usepackage{pgfplotstable}
\usepackage{indentfirst}
\usepackage{cancel}
\usepackage{xcolor}
\usepackage{listings}
\usepackage{multicol}
\newtheorem{teorema}{Teorema}
\renewcommand\lstlistingname{Código}
\renewcommand\lstlistlistingname{Código}

\lstset{language=R,
    basicstyle=\small\ttfamily,
    stringstyle=\color{green!70!black},
    otherkeywords={0,1,2,3,4,5,6,7,8,9},
    morekeywords={TRUE,FALSE},
    deletekeywords={data,frame,length,as,character},
    backgroundcolor =\color{blue!5!white} ,
    keywordstyle=\color{blue},
    commentstyle=\color{green!70!black},
}

\hypersetup{
	colorlinks=true,
	linkcolor=blue,
	citecolor=blue,
	urlcolor=blue
}

\renewcommand{\imprimircapa}{%
\begin{capa}%
\center
%\ABNTEXchapterfont\Large UNIVERSIDADE FEDERAL RURAL DE PERNAMBUCO\\
%USO DE EDITORES DE TEXTOS\\
%ATIVIDADE 5
%\vspace*{2cm}\\
\ABNTEXchapterfont\Large Ivanildo Batista \hspace{2cm} Marckis Lima
\vfill
\begin{center}
\ABNTEXchapterfont\bfseries\LARGE Regressão linear
\end{center}
\vfill
\large Recife-PE\\
\large \today
\vspace*{1cm}
\end{capa}
}

\titulo{Regressão Linear}
\autor{Ivanildo Batista \hspace{2cm} Marckis Lima}
\local{Recife-PE}
\data{\today}

\begin{document}

\imprimircapa

\pagenumbering{roman}

\listoffigures
\newpage

\listoftables
\newpage

\tableofcontents
\newpage

\textual

\pagenumbering{arabic}
\chapter{Introdução}
\label{ch:introducao}

Os pesquisadores estão na maioria das vezes interessados em encontrar relações entre uma variável e outras variáveis. Uma forma de encontrar essa relação é por meio da \textbf{análise de correlação} que é usada para quantificar ou medir a força ou grau de associação linear entre duas variáveis. A forma gráfica de verificar essa relação é por meio do \textbf{gráfico de dispersão}, que permite visualizar como uma variável se comporta em relação a outra. A medida mais comumente usada na análise de correlação é o coeficente de correlação (de \textit{Pearson}, de \textit{Spearman} ou de \textit{Kendall}). Entretanto outra forma de realizar dessa identificação é por meio da \textbf{análise de regressão}. 

O termo regressão foi usado pela primeira vez por Francis Galton em seu artigo \textit{Family likeness in stature}, onde ele observou que, embora pais altos tivesse filhos altos e pais baixos tivesse filhos baixos, a estatura média das crianças tendia a ``regredir" (daí dar o nome de regressão) para a média populacional. A análise de Galton visava saber se havia estabilidade na distribuição das alturas, entretanto a preocupação da análise moderna é descobrir como a altura média dos filhos varia, dada a altura dos pais. 

A análise de regressão trata do estudo de uma variável  que é dependente de outra ou outras variáveis (explanatórias ou independentes). O objetivo é prever/estimar o valor médio da variável dependente em termos dos valores das variáveis independentes, conforme \citeonline{gujarati2011econometria}. A regressão é o modelo matemático que relaciona a variável dependente com as variáveis independentes. Para \citeonline{angrist2008mostly} a regressão é um dispositivo computacional para estimar diferenças entre um grupo de controle e de tratamento em um experimento. \citeonline{andrade2019econometria} afirma que o modelo de regressão é o mais popular para estudar a relação entre variáveis e isso deve-se pela sua fácil aplicação, baixo custo computacional, interpretação simples, apresenta propriedades que são desejáveis e a maioria dos pacotes estatísticos incluem rotinas prontas para a estimação desse modelo.

A regressão pode ser usada em diversas áreas como \textbf{economia} (relação entre despesas de consumo pessoal e renda pessoal, descobrir a resposta da demanda por um produto frente a variação dos preços, relação entre salário e desemprego, etc), \textbf{saúde} (relação entre pressão ocular e idade, mortalidade e consumo de drogras lícitas ou ilícitas), \textbf{agronomia} (dependência do rendimento de uma plantação em relação à temperatura, à quantidade de chuva e de sol e à aplicação de fertilizantes), entre outras.

\citeonline{yan2009linear} explica que existem três tipos de regressão. A primeira é a \textbf{regressão linear simples}, que será o objeto de dsicussão nesse trabalho mais a frente. O segundo tipo de modelo de regressão é a \textbf{regressão linear múltipla}, que serve para para modelar a relação entre uma variável dependente $y$ com mais de uma variável independente ($x_1, x_2, \dots, x_p$). O modelo de regressão pode ser escrito conforme abaixo

\begin{equation}
    y = \beta_0 + \beta_1 x_1 + \dots + \beta_p x_p + \varepsilon
\end{equation}

o $y$ é a variável dependente, os coeficientes $\beta_0, \beta_1, \beta_2, \dots, \beta_p$ são os regressores, $x_1, x_2, \dots, x_p$ são as variáveis dependentes e $\varepsilon$ é o termo de erro, que assim como no modelo anterior, é normalmente distribuído com $E(\varepsilon) = 0$ e a variância é constante $Var(\varepsilon) = \sigma^2$. Notar que a regressão linear simples é um caso particular da regressão linear múltipla, quando $x_2 = x_3 = \dots = x_p = 0$.

\noindent O terceiro tipo de regressão é a \textbf{regressão não linear} a qual assume que a relação entre a variável dependente a as variáveis independentes é não linear nos parâmetros (diferentes das anteriores que se assume a linearidade dos parâmetros). Um exemplo desse tipo de regressão pode ser visto abaixo

\begin{equation}
    y = \dfrac{\alpha}{1 + e^{\beta t}} + \varepsilon
\end{equation}

\noindent Onde $y$ é o crescimento de um determinado organismo em função do tempo $t$, $\alpha$ e $\beta$ são os parâmetros do modelo e o $\varepsilon$ é o erro. Em termos de estimação dos parâmetros, seleção do modelo, diagnóstico, seleção de variáveis, detecção de \textit{outliers} e de observações influentes, os modelos não lineares são mais complicados. 

\noindent \citeonline{yan2009linear} ainda explica que o modelo de regressão possui três objetivos:

\begin{itemize}
    \item[1-] Estabelecer uma relação causal entre a variável dependente $y$ e os seus regressores $x_1, x_2, \dots, x_p$. 
    
    \item[2-] Prever o valor de $y$ com base no conjunto de valores de  $x_1, x_2, \dots, x_p$. É preciso definir se o modelo pode ser escrito conforme abaixo
    
    $$
    \text{Variável resposta} = \text{função do regressores} + \text{erro}
    $$
    
    ou apenas no formato matemático
    
    $$
    y = f(x_1,x_2,\dots,x_p) + \varepsilon
    $$
    \item[3-] Realizar uma seleção criteriosa dos regressores  $x_1, x_2, \dots, x_p$ para identificar quais as variáveis são mais importantes do que outras para explicar o comportamento da variável dependente $y$, para que a relação causal possa ser determinada de forma mais precisa e eficiente.
    
\end{itemize}

\noindent Nesse trabalho iremos abordar os conceitos por trás na regressão linear simples no \autoref{ch:modelo_regressao} , seus pressupostos no \autoref{ch:pressupostos}, formas de estimação dos parâmetros do modelo no \autoref{ch:estimacao}, avaliação do modelo no \autoref{ch:avaliacao_do_modelo}, inferência dos parâmetros no \autoref{ch:inferencia_dos_parametros} e, por fim, serão realizados exemplos no \autoref{ch:exemplos} com a linguagem \textit{R}.








\chapter{Modelo de regressão linear simples}
\label{ch:modelo_regressao}

\noindent A regressão linear simples (também chamada de regressão de primeiro grau) serve para modelar a relação linear entre duas variáveis. Uma deles é a variável dependente e outra é a variável independente.

Ela possui o nome de \textit{linear}, pois o valor da variável dependente é uma função linear da variável independente. Dessas variáveis são gerados os parâmetros da função de regressão, são esses parâmetros ($\beta's$) que definem a linearidade da função, pois eles são elevados a primeira potência. Sendo assim a regressão linear  simples tem como notação genérica a fórmula abaixo

\begin{equation}\label{eq:eq21}
    y = \beta_0 + \beta_1 x + \varepsilon
\end{equation}

\noindent A função acima possui cinco componentes, que são:

\begin{itemize}
    \item A variável dependente $y$ é variável que deseja-se saber seu comportamento, também é chamada de \textbf{variável alvo} ou \textbf{variável resposta}.
    \item A variável $x$ é a variável independente e será usada para explicar o comportamento da variável resposta, também é chamada de \textbf{preditor} ou \textbf{variável exploratória}. 
    \item Para a regressão linear simples, o $\beta_0$ é o intercepto do modelo ou o valor $y$, quando o valor de $x$ é igual a zero (variação média de $y$ quando $x$ não varia). Também é conhecido coomo \textbf{constante da regressão}.
    \item Para a regressão linear simples, o $\beta_1$ é a inclinação da reta de regressão. Ele é o \textbf{coeficiente de regressão} e representa a variação de $y$ em função da variação de uma unidade da variável $x$.
    \item $\varepsilon$ é o erro ou \textbf{resíduo}, diferença entre o valor observado de $y$ e o correspondente ponto da reta de regressão. Assume-se que tem valor esperado igual a zero, $E(\varepsilon) = 0$ e que a sua variância é constante, $Var(\varepsilon) = \sigma^2$. Pode ser interpretado como aquilo que o modelo não consegue explicar.
\end{itemize}

\noindent A \autoref{eq:eq21} também pode ser chamada de função de regressão populacional (FRP), entretanto as informações sobre uma população muitas vezes são desconhecidas, por conta disso usa-se amostras populacionais para estimar os parâmetros. Por conta disso os parâmetros são estimativas, então essa função deve ser representada como abaixo

\begin{equation}\label{eq:eq22}
    \hat{y} = \hat{\beta}_0 + \hat{\beta}_1 x + \hat{\varepsilon}
\end{equation}

\noindent Onde $\hat{y}$ é o estimador de $y$, $\hat{\beta}_0$ é o estimador de $\beta_0$, $\hat{\beta}_1$ é o estimador de $\beta_1$ e $\hat{\varepsilon}$ o erro estimado. Para essa função damos o nome de função de regressão amostral (FRA).

\section{Exemplo de regressão linear simples}

\noindent Na \autoref{fig:reta} temos um exemplo de uma reta de regressão com a função de regressão em sua legenda. Os dados usados são de consumo de cigarros e taxa de mortalidade, sendo obtidos do exemplo da introdução do livro \textit{Linear Regression Analysis} do \citeonline{yan2009linear}. Esses mesmos dados estão na \autoref{Tab: tabela}.

\begin{figure}[H]
\centering
\caption{Exemplo de reta de regressão}
\includegraphics[scale=.95]{imagens/imagens_cap2/regressao.PNG}
\label{fig:reta}
\end{figure}

\noindent Na figura vemos a reta de regressão na cor vermelha e as observações como pontos na cor preta. A reta de regressão é a reta que tem a menor distância para todas as observações. Na legenda da \autoref{fig:reta} há a equação da regressão

\begin{equation}\label{eq:eq22}
    y(x) = -2.89 + 1.09x
\end{equation}

\noindent Observa-se \autoref{fig:reta} que a correlação entre as variáveis é positiva: a medida que o valor do número de fumantes aumenta, o valor da taxa de mortaludade também aumenta. A vantagem da análise de regressão é trazer informações adicionais sobre essa relação : o $\hat{\beta}_0$ dessa equação é o valor de $-2.89$ e o $\hat{\beta}_1$ o valor de $1.09$. A interpretação dada a esse modelo é que a variação média de uma unidade de número de fumantes leva a um aumento (o valor da inclinação é positivo) médio de 1.09 unidade na taxa de mortalidade. \\

\noindent Essa função de regressão estimada permite saber para qual número de fumante a taxa de mortalidade média será igual a zero, $y(x) = 0$

\begin{align}
    & 0 = -2.89 + 1.09x \\
   & 2.89 = 1.09x \\
   & x = \dfrac{2.89}{1.09} = 2.65 \approx 3
\end{align}

\noindent Isolando o valor de $x$ temos que quando o número de fumantes for aproximadamente (arredondando de 2.65 para 3, já que não existem 2.65 pessoas), então a taxa de mortalidade média será aproximadamente zero.

\begin{table}[H]
\centering
\caption{Número de fumantes x Taxa de mortalidade}
\begin{tabular}{cllllllllll}
\hline
Número de fumantes & \multicolumn{1}{c}{77} & \multicolumn{1}{c}{112} & \multicolumn{1}{c}{137} & 113 & 117 & 110 & 94 & 125 & 116 & \multicolumn{1}{c}{\textbf{133}} \\
Mortalidade & \multicolumn{1}{c}{84} & \multicolumn{1}{c}{96} & \multicolumn{1}{c}{116} & 144 & 123 & 139 & 128 & 113 & 155 & \multicolumn{1}{c}{146} \\ \hline
Número de fumantes & 102 & 115 & 111 & 105 & 93 & 87 & 88 & 91 & 102 & 100 \\
Mortalidade & 101 & 128 & 118 & 115 & 113 & 79 & 104 & 85 & 88 & 120 \\ \hline
Número de fumantes & 91 & 76 & 104 & 66 & 107 &  &  &  &  &  \\
Mortalidade & 104 & 60 & 128 & 51 & 86 &  &  &  &  &  \\ \hline
\end{tabular}%
\label{Tab: tabela}
\end{table}

\noindent Podemos usar a \autoref{eq:eq22} para estimar os valores da taxa de mortalidade média para cada número de fumantes observado. Na \autoref{Tab: tabela2} temos os valores reais e os valores estimados da taxa de mortalidade.


\begin{table}[H]
\caption{Mortalidade observada x Mortalidade estimada}
\begin{tabular}{ccccccccccc}
\hline
Observada & 84 & 96 & 116 & 144 & 123 & 139 & 128 & 113 & 155 & 146 \\
Estimada & 81,04 & 119,2 & 146,44 & 120,3 & 124,6 & 117,01 & 99,6 & 133,4 & 123,6 & 142,1 \\ \hline
Observada & 101 & 128 & 118 & 115 & 113 & 79 & 104 & 85 & 88 & 120 \\
Estimada & 108,3 & 122,5 & 118,1 & 111,6 & 98,5 & 91,94 & 93,0 & 96,3 & 108,3 & 106,11 \\ \hline
Observada & 104 & 60 & 128 & 51 & 86 & \textbf{} & \textbf{} & \textbf{} & \textbf{} & \textbf{} \\
Estimada & 96,3 & 79,95 & 110,5 & 69,1 & 113,74 & \textbf{} & \textbf{} & \textbf{} & \textbf{} & \textbf{} \\ \hline
\end{tabular}%
\label{Tab: tabela2}
\end{table}

\noindent Por ser um modelo simples é normal que os valores não sejam exatos, mas pode-se observar valores estimados bem próximos dos reais, como por exemplo $ (84 - 81.04), (123 - 120.3), (146 -142.1), (118 - 118.1) $. E a diferença entre esses valores reais e os estimados (ou observados) será o nosso erro $\varepsilon$ ou \textbf{resíduo}.\\

\noindent Abaixo temos a \autoref{Tab:Tabela3} com os resíduos do nosso modelo que, no \autoref{ch:avaliacao_do_modelo}, serão fruto de análise.

\begin{table}[H]
\centering
\caption{Resíduos}
\begin{tabular}{llllllllll}
\hline
2,96 & -23,19 & -30,44 & 23,72 & -1,64 & 21,99 & 28,43 & -20,36 & 31,45 & 3,92 \\ \hline
-7,29 & 5,54 & -0,1 & 3,44 & 14,52 & -12,94 & 10,97 & -11,3 & -20,29 & 13,89 \\ \hline
7,7 & -19,95 & 17,53 & -18,05 & -27,74 & \textbf{} & \textbf{} & \textbf{} & \textbf{} & \textbf{} \\ \hline
\end{tabular}
\label{Tab:Tabela3}
\end{table}
\chapter{Pressupostos}
\label{ch:pressupostos}

\noindent O modelo de regressão linear baseia-se em várias suposições que determinam o quão bem ele opera. A maioria deles diz respeito às características dos dados populacionais e enfoca os erros de previsão ($\varepsilon_i$). Mas ter acesso às informações de uma população é pouco comum, então devemos avaliar, de forma aproximada ou indireta, as suposições do modelo de regressão linear com informações de uma amostra. Em outras palavras, como não temos informações da população, não podemos calcular $\varepsilon_i$ diretamente. A amostra inclui apenas de $x_i$ e de $y_i$, portanto, devemos usar uma estimativa de $\varepsilon_i$. Esta estimativa, descrita anteriormente como o termo de erro na \autoref{eq:eq21}, é representado pelos resíduos do modelo, que são calculados como ($y_i - \hat{y}_i$). Em vez de distinguir os erros de previsão da população e da amostra, no entanto, presumiremos que a amostra fornece uma boa estimativa de $Y_i$ (valores de $y_i$ para a população)  com $\hat{y_i}$, de modo que ($y_i - \hat{y}_i$) $\cong$ ($y_i - Y_i$).

\section{Suposições para o modelo}

\noindent As suposições necessárias para o Modelo de Regressão Linear são:

\begin{itemize}
    \item[i] O erro tem média zero e variância $\sigma ^2$ , desconhecida;
    \item[ii] Os erros são não correlacionados;
    \item[iii] Os erros têm distribuição normal;
    \item[iv] A variável regressora explicativa assume valores fixos.
\end{itemize}


\noindent As suposições $i$ e $iii$, simbolicamente, podem ser representadas por:  

\begin{equation}
\label{eq:erro}
    \varepsilon _i \sim N(0,\sigma^2)
\end{equation}

\section{Pressupostos da Análise de Regressão}

Para obtenção dos resultados, a análise de regressão baseia-se em quatro pressupostos básicos :

\subsection{Linearidade}

\noindent O valor médio de  $y_i$ é uma função em linha reta de $x_i$. Em outro palavras, $y_i$ e $x_i$ têm uma relação linear. Apesar de parecer um pressuposto restritivo matematicamente toda função não-linear pode ser transformada numa função linear através de técnicas logarítmicas, polinomiais e de relações recíprocas. Não nos cabe neste texto discutir as formulações matemáticas de transformação, porém a sua existência é de fundamental importância uma vez que a análise de regressão não pode ser aplicada se a função não puder ser transformada para a forma linear.

\subsection{Independência dos Resíduos}

\noindent Os erros de previsão ($\varepsilon_i$) são estatisticamente independentes um do outro. Na prática, isso muitas vezes implica que as observações são independentes. Uma maneira de (quase) garantir isso é para usar amostragem aleatória simples.
A violação do pressuposto da independência dos resíduos implica na existência de forte correlação (autocorrelação) entre os residuais sucessivos. Isto é, $e_t$ não é independente de $e_{t-1} \dots, e_{t-i+1} \dots, e_{t+1}, e_{t+2}, \dots, e_{t+n}$. A falta de independência não afeta o valor dos parâmetros estimados, mas afeta diretamente as variâncias estimadas. A falta de independência dos resíduos implica em $R^2$ e \textit{estatística} $F$ elevados e \textit{teste t} reduzido se a autocorrelação é positiva e todos os testes com resultados elevados se a autocorrelação for negativa. Na \autoref{fig:indep}, considerando os resíduos no eixo \textbf{y} e os valores treinados $\hat{y}_i$ no eixo \textbf{y}, vemos no primeiro gráfico que não há um padrão nos dados, mas no segundo gráfico já percebe-se a presença de uma tendência nos resíduos que é um comportamento de um modelo com autocorrelação (sem independência nos resíduos).

\begin{figure}[H]
\centering
\caption{Independência \textit{vs} Dependência}
\includegraphics[scale=.75]{imagens/imagens_cap3/independencia.PNG}
\label{fig:indep}
\end{figure}

\subsection{Homocedasticidade (ou variância constante)}

\noindent Os erros de previsão têm variância equivalente para todos os valores possíveis de $x_i$. Em outras palavras, a variância dos erros é considerada constante ao longo da distribuição de $x_i$. Neste ponto, pode ser mais simples, embora impreciso, pensar sobre os valores de $y_i$ e pergunte se sua variabilidade é equivalente em diferentes valores de $x_i$. Se os resíduos não estão distribuídos ao longo da linha de regressão em torno de todo o intervalo de observações, o pressuposto da variância constante, ou homocedasticidade, é violado. A \autoref{fig:reta} a seguir ilustra o significado da variância constante dos resíduos:

\begin{figure}[H]
\centering
\caption{Variância Constante dos Resíduos}
\includegraphics[scale=.65]{imagens/imagens_cap3/homo1.png}
\label{fig:reta}
\end{figure}

A  ocorrência  de  variâncias  não  constantes  nos  resíduos  é  chamada  de \textbf{heterocedasticidade}.  Sua  ocorrência  pode  estar  condicionada  a  especificações incorretas  no  modelo  de  regressão,  e  sua  detecção  é  possível  através  do  estudo residual dos erros. Na \autoref{fig:homoxhetero} vemos na primeira imagem um exemplo de homocedasticidade onde as variância do modelo é constante. Entretanto na segunda imagem observa-se que a medida que os valores de $x$ vão aumentando a variância também vai aumentando, logo há presença de heterocedasticidade.

\begin{figure}[H]
\centering
\caption{Homocedasticidade \textit{vs} Heterocedasticidade}
\includegraphics[scale=.8]{imagens/imagens_cap3/homoxhetero.png}
\label{fig:homoxhetero}
\end{figure}

\noindent Outra forma mais intuitiva de entender o problema de encontra-se na \autoref{fig:homoxhetero2}. Na primeira imagem, para cada observação de $x$ as distribuições são iguais, já na segunda as distribuições são diferentes.

\begin{figure}[H]
\centering
\caption{Homocedasticidade \textit{vs} Heterocedasticidade (densidade)}
\includegraphics[scale=.6]{imagens/imagens_cap3/homoxhetero2.png}
\label{fig:homoxhetero2}
\end{figure}


\noindent O  teste  \textit{Durbin-Watson}  pode  identificar a presença ou não de heterodasticidade  e sua  correção  esta vinculada  à  eliminação  de  algumas  variáveis  ou  a  transformação  matemática  do modelo,   trazendo   uniformidade   dos   erros   percentuais   ao   longo   da   linha   de regressão.
	
\subsection{Normalidade dos Resíduos}

\noindent Os erros são uma variável aleatória normalmente distribuída. Também assumimos que os erros têm uma média igual a zero na população, embora isso não seja especialmente importante. Simbolicamente, a suposição de normalidade é freqüentemente apresentada como e na \autoref{eq:erro}. A porção de variância da equação ($\sigma^2$) tem implicações para a suposição de homocedasticidade. Esta  hipótese  também  apresenta  características  pouco  restritivas  uma  vez que  os  resíduos  são  resultantes  de  um  sem  número  de  fatores  menos  importantes no  que  tange  a  influência  no  comportamento  da  variável  dependente  (senão deveriam  ser  incluídos  na  equação  de  regressão,  perdendo  sua  característica residual). Na média, sua influência pode ser desprezada, uma vez que o erro médio apresenta um comportamento “normalizado”.

Estatisticamente  se  possuímos  um  número  de  observações  superior  a  30  a previsão  de  dados assume a “normalidade”.  Isto porque  a  distribuição amostral dos estimadores    pode  ser  aproximada  a  curva  normal  onde n  possua  amplitude suficiente, o que na maior parte ocorre quando n é igual a 30. O Teorema do Limite Central  da  estatística  permite  esta  aproximação  e  torna  possível  o  uso  da  curva normal na avaliação da dispersão dos dados, inclusive dos resíduos, da amostra em torno  do  parâmetro  central  (média).  Assim  ao  calcularmos  sua  média  e  variância,  a extensão  de  possíveis  erros  pode  ser  avaliada;  o  que  introduz  um  intervalo  de confiança de 30 observações para a variância.

Quando  o  pressuposto  da  normalidade  dos  resíduos  é  questionado, pode-se realizar dois tipos de análise:

\begin{itemize}
    \item \textbf{Visual} : os  resíduos podem ser plotados com vistas a detecção de sua distribuição próxima a normal por meio de um histograma ou de um \textit{QQplot}, com o seu intervalo de variação (o maior menos o menor valor) pode ser medido com vistas a  determinação  de  sua  dispersão  (se  próxima  a  6.0  é  considerado  dentro  da distribuição normal).
    
    \item \textbf{Testes estatísticos}: Podem ser realizados os testes de \textit{Kolmogorov-Smirnov (K-S)}, \textit{K-S} corrigido de \textit{Lilliefors}, \textit{Shapiro-Wilk}, \textit{Anderson-Darling}, \textit{Cramer-von Mises}, teste de assimetria de \textit{D’Agostino}, teste de curtose de \textit{Anscombe-Glynn}, teste omnibus de \textit{D’Agostino-Pearson} e o \textit{Jarque-Bera}. \citeonline{yazici2007comparison} compara diversos testes de normalidade.
\end{itemize}
\chapter{Estimação dos parâmetros}
\label{ch:estimacao}

\noindent O primeiro passo, na análise de regressão, é obter as estimativas $\hat{\beta}_0$ e $\hat{\beta}_1$ dos
parâmetros $\beta_0$ e $\beta_1$ da regressão. Os valores dessas estimativas serão obtidos a partir de uma amostra de $n$ pares de valores $x_i$, $y_i$ (com $i = 1, 2, ..., n$), que correspondem a $n$ pontos num gráfico.\\

\noindent Obtemos, então:

$$\hat{y}_i = \hat{\beta}_0 + \hat{\beta}_1 x_i$$

\noindent onde $\hat{y}_i$, $\hat{\beta}_0$ e $\hat{\beta}_1$ são, respectivamente estimativas de $E(y_i) =\beta_0 + \beta_1 x_i$, $\beta_0$ e $\beta_1$. Para cada par de valores $x_i$, $y_i$ podemos estabelecer o desvio $$\varepsilon_i = y_i-\hat{y_i} = y_i-(\beta_0 + \beta_1 x_i)$$

\section{Estimação pelo método de mínimos quadrados ordinários}

\noindent O método dos mínimos quadrados para o modelo de regressão linear simples consiste em adotar como estimativas dos parâmetros $\beta_0$ e $\beta_1$ os valores que minimizam a soma dos quadrados dos desvios.
A partir da resposta real $y_i$ e da resposta prevista $\hat{y}_i = \beta_0 + \beta_1 xi$ atinge-se o mínimo entre todas as escolhas possíveis de coeficientes de regressão $\beta_0$ e $\beta_1$, ou seja, temos a função $Z$

$$Z = \sum_{i=1}^{n}\varepsilon_i^2 = \sum_{i=1}^{n}[y_i -(\beta_0 + \beta_1 x_i)]^2$$

\noindent Então queremos minimizar essa função

$$(\beta_0, \beta_1) = \text{min}_{\beta_0, \beta_1} \sum_{i=1}^{n} Z$$

\noindent ou

\begin{equation}
\label{eq:somaerro}
    (\beta_0, \beta_1) = \text{min}_{\beta_0, \beta_1} \sum_{i=1}^{n} [y_i -(\beta_0 + \beta_1 x_i)]^2
\end{equation}

\noindent Para realizar a estimação, vamos expandir a função $Z$

$$ Z = \sum_{i=1}^{n} (y_i - \beta_0 - \beta_1 x_i)(y_i - \beta_0 - \beta_1 x_i)$$

$$Z = \sum_{i=1}^{n} (y^2_i - 2\beta_0 y_i - 2\beta_1 y_i x_1 + 2\beta_0 \beta_1 x_i + \beta_0^{2} + \beta_1^{2} x_i^{2})$$

$$Z = \sum_{i=1}^{n} y^2_i - 2 \beta_0 \sum_{i=1}^{n}  y_i - 2 \beta_1 \sum_{i=1}^{n}  y_i x_1 + 2 \beta_0 \beta_1 \sum_{i=1}^{n}  x_i +  \sum_{i=1}^{n} \beta_0^{2} + \beta_1^{2} \sum_{i=1}^{n}  x_i^{2}$$

$$Z = \sum_{i=1}^{n} y^2_i - 2 \beta_0 \sum_{i=1}^{n}  y_i - 2 \beta_1 \sum_{i=1}^{n}  y_i x_i + 2 \beta_0 \beta_1 \sum_{i=1}^{n}  x_i +  n \beta_0^{2} + \beta_1^{2} \sum_{i=1}^{n}  x_i^{2}$$

$$Z = \sum_{i=1}^{n} y^2_i + 2 \beta_0 \beta_1 \sum_{i=1}^{n}  x_i +  n \beta_0^{2} + \beta_1^{2} \sum_{i=1}^{n}  x_i^{2} - 2 \beta_0 \sum_{i=1}^{n}  y_i - 2 \beta_1 \sum_{i=1}^{n}  y_i x_i$$

\noindent A função $Z$ terá mínimo solucionando as suas derivadas parciais em relação a $\beta_0$ e $\beta_1$ forem nulas. Assim temos que


\begin{equation}
\label{eq:za}
\frac{\partial Z}{\partial \beta_0} = 2n \beta_0 + 2 \beta_1 \sum_{i=1}^{n}  x_i  - 2 \sum_{i=1}^{n}  y_i = 0
\end{equation}

\begin{equation}
\label{eq:zb}
\frac{\partial Z}{\partial \beta_1} = 2 \beta_0 \sum_{i=1}^{n}  x_i + 2 \beta_1 \sum_{i=1}^{n}  x_i^{2} - 2 \sum_{i=1}^{n}  y_i x_i = 0
\end{equation}

\noindent Por se tratar de uma soma de quadrados de desvios, o ponto extremo será necessariamente um ponto de mínimo da função. Formalmente, pode-se verificar que as condições de segunda ordem para mínimo são satisfeitas. Simplificando \autoref{eq:za} e \autoref{eq:zb}, obtemos as equações normais das condições de primeira ordem: Para a \autoref{eq:za}

$$ 2n \beta_0 + 2 \beta_1 \sum_{i=1}^{n}  x_i  - 2 \sum_{i=1}^{n}  y_i = 0 $$

$$ 2n \beta_0 + 2 \beta_1 \sum_{i=1}^{n}  x_i  = 2 \sum_{i=1}^{n}  y_i$$

$$ 2 \sum_{i=1}^{n}  y_i = 2n \beta_0 + 2 \beta_1 \sum_{i=1}^{n}  x_i$$

$$ \boxed{\sum_{i=1}^{n}  y_i = n \beta_0 + \beta_1 \sum_{i=1}^{n}  x_i}$$

\noindent  Para a \autoref{eq:zb}

$$ 2 \beta_0 \sum_{i=1}^{n}  x_i + 2 \beta_1 \sum_{i=1}^{n}  x_i^{2} - 2 \sum_{i=1}^{n}  y_i x_i = 0 $$

$$ 2 \beta_0 \sum_{i=1}^{n}  x_i + 2 \beta_1 \sum_{i=1}^{n}  x_i^{2} = 2 \sum_{i=1}^{n}  y_i x_i$$

$$ \boxed{\beta_0 \sum_{i=1}^{n}  x_i + \beta_1 \sum_{i=1}^{n}  x_i^{2} = \sum_{i=1}^{n}  y_i x_i}$$


\noindent Chegamos ao sistema de equações normais

\begin{equation}
    \begin{cases}
      \sum_{i=1}^{n}  y_i = n \beta_0 + \beta_1 \sum_{i=1}^{n}  x_i\\
      \beta_0\sum_{i=1}^{n}  x_i + \beta_1 \sum_{i=1}^{n}  x_i^{2} = \sum_{i=1}^{n}  y_i x_i
    \end{cases}\,.
\end{equation}


\noindent Resolvendo o sistema, obtemos: 

\begin{equation}
\label{eq:eq43}
\beta_0 = \dfrac{\sum_{i=1}^{n}  y_i - \beta_1 \sum_{i=1}^{n}  x_i}{n}\\ \textbf{ ou }\\ \beta_0 = \dfrac{\sum_{i=1}^{n}  y_i}{n} - \beta_1 \dfrac{\sum_{i=1}^{n}  x_i}{n}
\end{equation}

\noindent Substituindo em

$$\beta_0\sum_{i=1}^{n}  x_i + \beta_1 \sum_{i=1}^{n}  x_i^{2} = \sum_{i=1}^{n}  y_i x_i $$

\noindent Temos

$$\bigg(\dfrac{\sum_{i=1}^{n}  y_i - \beta_1 \sum_{i=1}^{n}  x_i}{n}\bigg)\sum_{i=1}^{n}  x_i + \beta_1 \sum_{i=1}^{n}  x_i^{2} = \sum_{i=1}^{n}  y_i x_i $$

$$ \bigg(\dfrac{\sum_{i=1}^{n}  y_i \sum_{i=1}^{n}  x_i - \beta_1 (\sum_{i=1}^{n}  x_i)^2}{n}\bigg) +  \beta_1 \sum_{i=1}^{n}  x_i^{2} = \sum_{i=1}^{n}  y_i x_i$$

$$\beta_1  \sum_{i=1}^{n}  x_i^{2} - \dfrac{\beta_1}{n} \bigg(\sum_{i=1}^{n}  x_i \bigg)^2  = \sum_{i=1}^{n}  y_i x_i - \bigg( \dfrac{\sum_{i=1}^{n}  y_i \sum_{i=1}^{n}  x_i}{n}\bigg)$$

\noindent Multiplicando ambos os lados por $n$

$$n\beta_1  \sum_{i=1}^{n}  x_i^{2} - \beta_1 \bigg(\sum_{i=1}^{n}  x_i\bigg)^2  = n\sum_{i=1}^{n}  y_i x_i - \sum_{i=1}^{n}  y_i \sum_{i=1}^{n}  x_i$$

\noindent Isolando o $\beta_1$

$$\beta_1  \bigg[n\sum_{i=1}^{n}  x_i^{2} - \bigg(\sum_{i=1}^{n}  x_i\bigg)^2 \bigg]   = n\sum_{i=1}^{n}  y_i x_i - \sum_{i=1}^{n}  y_i \sum_{i=1}^{n}  x_i$$

\begin{equation}
\label{eq:eq41}
\beta_1   = \dfrac{n\sum_{i=1}^{n}  y_i x_i - \sum_{i=1}^{n}  y_i \sum_{i=1}^{n}  x_i}{n\sum_{i=1}^{n}  x_i^{2} - \bigg(\sum_{i=1}^{n}  x_i\bigg)^2}
\end{equation}

\noindent Notar que 

$$ n\sum_{i=1}^{n}  y_i x_i - \sum_{i=1}^{n}  y_i \sum_{i=1}^{n}  x_i = n \bigg( \sum_{i=1}^{n} (y_i - \overline{y})(x_i - \overline{x}) \bigg)  $$

\noindent e que 

$$ n\sum_{i=1}^{n}  x_i^{2} - \bigg(\sum_{i=1}^{n}  x_i\bigg)^2 = n \bigg(  \sum_{i=1}^{n} (x_i - \overline{x})^2 \bigg)$$

\noindent Portanto

\begin{equation}
\beta_1 = \dfrac{n \bigg( \sum_{i=1}^{n} (y_i - \overline{y})(x_i - \overline{x}) \bigg)}{n \bigg(  \sum_{i=1}^{n} (x_i - \overline{x})^2 \bigg)} = 
\dfrac{\sum_{i=1}^{n} (y_i - \overline{y})(x_i - \overline{x}) }{\sum_{i=1}^{n} (x_i - \overline{x})^2}
\end{equation}

\noindent Dado que encontramos o $\beta_1$, podemos encontrar o $\beta_0$ . Anteriormente encontramos que 
\begin{equation}
\label{eq:eq45}
    \beta_0 = \dfrac{\sum_{i=1}^{n}  y_i - \beta_1 \sum_{i=1}^{n}  x_i}{n}
\end{equation}

\noindent Substituindo o $\beta_1$ da \autoref{eq:eq41} na \autoref{eq:eq43}, temos

$$ \beta_0 = \dfrac{\sum_{i=1}^{n}  y_i}{n} - \beta_1 \dfrac{\sum_{i=1}^{n}  x_i}{n} $$

$$ \beta_0 = \dfrac{\sum_{i=1}^{n}  y_i}{n} - \Bigg[ \dfrac{n\sum_{i=1}^{n}  y_i x_i - \sum_{i=1}^{n}  y_i \sum_{i=1}^{n}  x_i}{n\sum_{i=1}^{n}  x_i^{2} - \bigg(\sum_{i=1}^{n}  x_i\bigg)^2} \Bigg] \dfrac{\sum_{i=1}^{n}  x_i}{n} $$

$$ n\beta_0 = \sum_{i=1}^{n}  y_i - \Bigg[ \dfrac{n\sum_{i=1}^{n}  y_i x_i - \sum_{i=1}^{n}  y_i \sum_{i=1}^{n}  x_i}{n\sum_{i=1}^{n}  x_i^{2} - \bigg(\sum_{i=1}^{n}  x_i\bigg)^2} \sum_{i=1}^{n}\Bigg]  x_i $$

$$ n\beta_0 = \dfrac{n \sum_{i=1}^{n} y_i \sum_{i=1}^{n} x_i^{2} -  \cancel{\sum_{i=1}^{n} y_i (\sum_{i=1}^{n} x_i)^2} - n \sum_{i=1}^{n} y_i x_i \sum_{i=1}^{n} x_i + \cancel{\sum_{i=1}^{n} y_i (\sum_{i=1}^{n} x_i)^2}}{n \sum_{i=1}^{n} x_i^{2} - (\sum_{i=1}^{n} x_i)^2} $$

$$ \cancel{n}\beta_0 = \dfrac{\cancel{n} \sum_{i=1}^{n} y_i \sum_{i=1}^{n} x_i^{2} - \cancel{n} \sum_{i=1}^{n} y_i x_i \sum_{i=1}^{n} x_i }{n \sum_{i=1}^{n} x_i^{2} - (\sum_{i=1}^{n} x_i)^2} $$

\begin{equation}
\label{eq:eq42}
\boxed{\beta_0 = \dfrac{ \sum_{i=1}^{n} y_i \sum_{i=1}^{n} x_i^{2} - \sum_{i=1}^{n} y_i x_i \sum_{i=1}^{n} x_i }{n \sum_{i=1}^{n} x_i^{2} - (\sum_{i=1}^{n} x_i)^2}}
\end{equation}

\noindent Pode-se notar também que da \autoref{eq:eq45}, $\dfrac{\sum_{i=1}^{n}  y_i}{n}$ é  média de $y$ ($\overline{y}$) e $\dfrac{\sum_{i=1}^{n}  x_i}{n}$ é a média de $x$ ($\overline{x}$), então $\beta_0$ pode escrito como

\begin{equation}
\label{eq:eq46}
\boxed{\beta_0 = \overline{y} - \beta_1 \overline{x}}
\end{equation}

As variâncias dos estimadores $\beta_0$ e $\beta_1$ são dadas, respectivamente, por

$$ Var(\hat{\beta}_0) = \sigma^2 \bigg[ \dfrac{\sum_{i=1}^{n} x_i^2}{n \sum_{i=1}^{n} (x_i - \overline{x})^2} \bigg]$$

$$ Var(\hat{\beta}_1) =  \dfrac{\sigma^2}{n \sum_{i=1}^{n} (x_i - \overline{x})^2}$$

A covariância entre os parâmetros é dada por

$$ Cov(\hat{\beta}_0,\hat{\beta}_1) = -\sigma^2 \bigg[ \dfrac{\overline{x}}{n \sum_{i=1}^{n} (x_i - \overline{x})^2} \bigg]$$

\section{Estimador para a variância}

\noindent Após encontrar os estimadores para $\beta_0$ e $\beta-1$ é preciso encontrar um estimador para a variância dos erros do modelo. A variância $\sigma^2 = Var(\varepsilon_i) = E(\varepsilon_i^2)$. Podemos usar a variância amostral dos resíduos como um estimador para a variância populacional dos erros.

\noindent Ao fazer isso, incorporaremos uma correção de graus de liberdade (número de observações menos o número de parâmetros). Temos, assim, o seguinte estimador da variância dos erros:

\begin{equation}
    \label{eq:eq47}
        \hat{\sigma}^2 = \dfrac{1}{n - p} \sum_{i=1}^{n} (y_i - \hat{y}_i)^2
\end{equation}

\noindent Como o modelo de regressão linear simples tem dois parâmetros, então

\begin{equation}
    \label{eq:eq48}
        \boxed{\hat{\sigma}^2 = \dfrac{1}{n - 2} \sum_{i=1}^{n} (y_i - \hat{y}_i)^2}
\end{equation}

\noindent Uma relação bastante importante, para futuramente realizar inferência sobre a variância estimada, é 

\begin{equation}
    \label{eq:eq49}
    (n-2)\dfrac{\hat{\sigma}^2}{\sigma^2}
\end{equation}

\noindent Podemos ver que é a relação entre a variância estimada e uma variância dada, multiplicada pelos graus de liberdade. Vamos citá-la, mas iremos usá-la mais a frente.

\section{Propriedades dos estimadores}

\noindent Agora vamos ver algumas propriedades estatísticas do modelo de mínimos quadrados ordinários considerando a suposição que $E(\varepsilon_i) = 0$, $Var(\varepsilon) = \sigma^2$ e os erros $\varepsilon_i's$ para $i=1,2,\dots$ são independentes. Abaixo seguem os principais teoremas

\begin{teorema}
O estimador de mínimos quadrados $\hat{\beta}_0$ é não viesado para $\beta_0$.
\end{teorema}

\begin{teorema}
O estimador de mínimos quadrados $\hat{\beta}_1$ é não viesado para $\beta_1$.
\end{teorema}

\begin{teorema}
$Var(\hat{\beta}_1) = \dfrac{\sigma^2}{nSQ_{Total}}$
\end{teorema}

\begin{teorema}
Os estimadores de mínimos quadrados $\hat{\beta}_1$ e $\overline{y}$ não estão correlacionados. Sob a suposição de normalidade de $y_i$ para $i = 1, 2,\dots, n$, $\hat{\beta}_1$ e $\overline{y}$ são normalmente distribuídos e independentes.
\end{teorema}

\begin{teorema}
$Var(\hat{\beta}_0) = \bigg( \dfrac{1}{n} + \dfrac{\overline{x}^2}{nSQ_{Total}} \bigg) $
\end{teorema}

\section{Exemplo de estimação dos parâmetros}

\noindent Utilizando o exemplo da \autoref{Tab: tabela} do \autoref{ch:modelo_regressao} podemos calcular os componentes da fórmula \autoref{eq:eq41} e \autoref{eq:eq42}, que seguem abaixo já calculados.

$$ \sum_{i=1}^{n} x_i = 2572 \qquad \sum_{i=1}^{n} y_i = 2725$$

$$ \sum_{i=1}^{n} x_i^2 = 271706 \qquad \sum_{i=1}^{n} y_i x_i = 288068$$

$$n=25$$

\noindent Substituindo os valores na \autoref{eq:eq41}

$$\beta_0 =  \dfrac{(271706 \cdot 2725) - (2572 \cdot 288068)}{25 \cdot 271706 - (2572)^2} = \dfrac{740398850 - 740910896}{6792650 - 6615184}  = -\dfrac{512046}{177466}$$

$$\beta_0 = -2.88531 \text{ ou } \boxed{\beta_0 \approx -2.89}$$

\noindent Temos que o $a$ (ou $\beta_0$) é o mesmo encontrado no exemplo do \autoref{ch:modelo_regressao}. Agora para o valor de $b$, conforme a \autoref{eq:eq42}.

$$\beta_1 = \dfrac{25 \cdot 288068 - (2572 \cdot 2725)}{25 \cdot 271706 - (2572)^2} = \dfrac{7201700 - 7008700}{6792650 - 6615184} = \dfrac{193000}{177466}$$

$$\beta_1 = 1.0875 \text{ ou } \boxed{\beta_1 \approx 1.09}$$

\noindent Estimando os parâmetros manualmente chega-se no mesmo resultado do exemplo, \autoref{eq:eq22} $y = -2.89 + 1.09x$. No \autoref{ch:exemplos} iremos realizar esse procedimento no \textit{software R}. 





\chapter{Avaliação do modelo}
\label{ch:avaliacao_do_modelo}

\noindent No processo de seleção de covariáveis, diferentes critérios podem ser
usados para comparar os modelos produzidos. Alguns deles são
descritos na sequência.

\section{Coeficiente de Determinação - \textit{$R^2$}}
\label{ch:r2}

\noindent Queremos saber o quão bem ajustado o nosso modelo de regressão linear simples está ou o quão bem a linha reta de regressão está ajustada aos dados. Não existe um ajuste perfeito dos dados, raramente isso acontece; mas queremos que pelo menos os resíduos em torno da média sejam os menores possíveis. Para sabe a ``qualidade" desse ajuste usamos o coeficiente de determinação $R^2$ que é uma medida que resume o quão ajustados os dados estão em relação a linha de regressão.\\

\noindent Para calcular o $R^2$ para o modelo de regressão linear simples partimos de que

\begin{equation}
    y_i = \hat{y}_i + \hat{\varepsilon}_i
\end{equation}

\noindent Subtraindo ambos os lados por $\overline{y}$ temos

\begin{equation}
    y_i - \overline{y} = \hat{y}_i - \overline{y} + \hat{\varepsilon}_i
\end{equation}

\noindent Definindo $(y_i - \overline{y}) = \text{y}$ e $(\hat{y}_i - \overline{y}) = \hat{\text{y}}$, então

\begin{equation}
    \label{eq:eq51}
    \text{y} = \hat{\text{y}} + \hat{\varepsilon}_i
\end{equation}


\noindent Elevando ambos os lados da \autoref{eq:eq51} e somando na amostra temos

\begin{equation}
    \sum_{i=1}^{n} \text{y}^2 = \sum_{i=1}^{n} \hat{\text{y}}^2 + \sum_{i=1}^{n} \hat{\varepsilon}_i^2 + 2 \sum_{i=1}^{n} \hat{\text{y}} \hat{\varepsilon}_i
\end{equation}

\noindent Como $\sum_{i=1}^{n} \hat{\text{y}} \hat{\varepsilon}_i = 0$

\begin{equation}
    \sum_{i=1}^{n} \text{y}^2 = \sum_{i=1}^{n} \hat{\text{y}}^2 + \sum_{i=1}^{n} \hat{\varepsilon}_i^2
\end{equation}

\noindent Então,

\begin{equation}
\label{eq:eq52}
    \sum_{i=1}^{n} (y_i - \overline{y})^2 = \sum_{i=1}^{n} (\hat{y}_i - \overline{y})^2 + \sum_{i=1}^{n} \hat{\varepsilon}_i^2
\end{equation}

\noindent O componente $\sum_{i=1}^{n} (y_i - \overline{y})^2$, também representado por $SQ_{Total}$ (soma do quadrados totais) é a variabilidade total dos dados (corrigida pela média). O componente $\sum_{i=1}^{n} (\hat{y}_i - \overline{y})^2$ ou $SQ_{Reg}$ (soma dos quadrados da regressão) é variabilidade dos dados explicada pela regressão; e o componente $\sum_{i=1}^{n} \hat{\varepsilon}_i^2$ ou $SQ_{Res}$ é a soma do quadrado dos resíduos, variabilidade que o modelo de regressão não consegue explicar. Então a \autoref{eq:eq52} pode ser escrita como

\begin{equation}
\label{eq:eq53}
    SQ_{Total} = SQ_{Reg} + SQ_{Res}
\end{equation}

\noindent Como dito anteriormente o coeficiente de determinação $R^2$ corresponde à proporção da variação dos dados explicada pela regressão:

$$R^2=\frac{SQ_{Reg}}{SQ_{Total}}= 1 - \frac{SQ_{Res}}{SQ_{Total}}$$

\noindent Características do coeficiente de determinação $R^2$

\begin{itemize}

\item Notar que $SQ_{Total} > 0$ e que o $SQ_{Reg} \geq 0$. Isso implica que o $R^2 \geq 0$. Note também que $SQ_{Reg} \geq SQ_{Total}$, implicando que $R^2 \geq 1$. Ou seja, $\boxed{0 \geq R^2 \geq 1}$.

\item O $R^2$ é o coeficiente de correlação linear entre $y_i$ e $\hat{y}_i$.

\item Quanto maior o $R^2$, maior o poder explicativo do modelo.

\item é uma proporção. Ele mede a proporção da variação da resposta em torno de sua média amostral que pode ser explicada usando o modelo de regressão ao invés do modelo simples $y_i = \beta_0 + \varepsilon$.

\item O que garante que o $R^2$ varie entre 0 e 1 é a presença do intercepto no modelo, caso o modelo não possua intercepto deve usar o $R^2$ não centrado ($\tilde{R^2}$).

\item $R^2$ é não decrescente, ou seja, a inserção de variáveis no modelo nunca o fará decresce (limitação).

\item O coeficiente de determinação não é apropriado para comparar modelos com diferentes números de parâmetros, uma vez que $R^2$ sempre aumenta com a inclusão de novas covariáveis.
\end{itemize}

\section{Coeficiente de Determinação Ajustado - $R^2_{Aj}$}
\label{ch:r2a}

\noindent Como dito anteriomente, o coeficiente de determinação possui a limitação de ser não decrescente. Entretanto o \textbf{coeficiente de determinação ajustado},$R^2_{\text{ajustado}}$, não sofre dessa limitação. O $R^2_{\text{ajustado}}$ é definido por:

\begin{equation}
    \label{eq:eq53}
    1 - \dfrac{SQ_{Reg}/(n-p)}{SQ_{Total}/(n-1)}
\end{equation}

\noindent Estabelecendo uma relação entre $R^2_{\text{ajustado}}$ e o $R^2$

$$R^2_{Aj}= 1 -\left ( \dfrac{n-1}{n-p} \right ) \bigg(\dfrac{SQ_{Reg}}{SQ_{Total}}\bigg)$$

$$R^2_{Aj}= 1 -\bigg( \frac{n-1}{n-p} \bigg) \left(1-R^2\right)$$

\noindent em que $n$ e $p$ são o número de observações e o número de parâmetros do modelo.

\begin{itemize}
\item Diferentemente do que ocorre para $R^2$, o valor de $R^2_{Aj}$ pode não
aumentar mediante inclusão de novas variáveis ao modelo. Deve-se optar por modelos com maiores valores de $R^2_{Aj}$.

\item $R^2_{\text{ajustado}} \leq R^2$.

\item O $R^2_{\text{ajustado}}$ pode ser menor que zero (negativo).
\end{itemize}

\section{Quadrado Médio de Resíduos}

\noindent Definido por:

$$QM_{Res}= \frac{SQ_{Res}}{n-p}=\frac{\sum_{i=1}^{n}(y_i-\widehat{y}_i)^2}{n-p}$$

\noindent também pode ser usado para comparação e seleção de modelos de regressão
\begin{itemize}
\item Deve-se optar por modelos com menores valores para $QM_{Res}$;
\item Pode-se mostrar que minimizar $QM_{Res}$ é equivalente a maximizar $R^2_{Aj}$, de forma que os dois critérios conduzem à seleção do mesmo conjunto
de covariáveis.
\end{itemize}

\section{$C_p$ de \textit{Mallows}}

\noindent O coeficiente $C_p$ de \textit{Mallows} \cite{mallows2000some} é definido por:

$$\frac{1}{\sigma^2}\sum_{i=1}^{n}E[\widehat{y}_i-E(y_i)]^2$$

\noindent podendo ser estimado por:

$$C_p=\frac{SQ_{Res}}{\sigma^2}+2p-n$$

\noindent em que $\sigma^2$ é dado pelo quadrado médio de resídus do modelo que inclui todas as covariáveis.

\begin{itemize}
\item Para o modelo completo, com $p$ parâmetros, $C_p = p$.
\item Para submodelos, definidos por subconjunto das covariáveis, menores
valores para $C_p$ são preferíveis.
\item Uma estratégia para selecionar modelos com base nos valores de $C_p$ é
plotar $C_p$ versus $p$ e adicionar ao gráfico a reta $C_p = p$.
\item Modelos com viés reduzido terão pontos próximos a reta.
\item Dentre os modelos com pontos próximos à reta, deve-se optar por
aquele com menor $C_p$ (e $p$, consequentemente).

\end{itemize}

\section{Estatística \textit{PRESS}}

\noindent A Estatística \textit{PRESS} \cite{allen1971prediction} permite  avaliar a qualidade preditiva dos modelos de regressão, sendo definida por:

$$PRESS=\sum_{i=1}^{n}[y_i-\widehat{y}_i]^2=\sum_{i=1}^{n}\left ( \frac{r_i}{1-h_{ii}} \right )^2$$

em que $\widehat{y}_i$ é obtido com base no modelo ajustado apenas com as demais
$n-1$ observações $(i = 1, 2, ..., n)$.

\begin{itemize}
\item Menores valores da estatística PRESS indicam modelos com maior poder preditivo.
\end{itemize}

\section{Critérios de informação}

\noindent \textbf{Critério de informação} são métricas que mensuram a qualidade de um modelo estatístico visando também a sua simplicidade. Fornece, portanto, uma métrica para comparação e seleção de modelos, em que menores valores do critério escolhido representa uma maior qualidade e simplicidade.

\subsection{Critério de Informação de Akaike - AIC}

\noindent O critério de informação de Akaike ou \textit{Akaike Information Criterion} \cite{akaike1974new}, ou
simplesmente \textit{AIC}, é definido por:

$$AIC=-2 log(\widehat{\theta})+2p$$

\noindent em que $log(\widehat{\theta}$ é a log-verossimilhança maximizada do modelo (calculada com base nos emv’s dos parâmetros) e $p$ o número de parâmetros.

\begin{itemize}
\item O AIC pode ser usado para qualquer modelo ajustado por máxima verossimilhança. No caso de um modelo de regressão linear temos:
\end{itemize}

$$AIC=-n log(SQ_{Res}/n)+2p$$

\begin{itemize}
\item O componente $2p$, na expressão do $AIC$, atua como termo de penalização atribuído à complexidade (número de parâmetros) do modelo.
\end{itemize}

\subsection{Critério de Informação Bayesiano - BIC}

\noindent Um critério alternativo ao AIC é o Critério de Informação Bayesiano ou BIC \cite{schwarz1978estimating}, definido, para um modelo de regressão linear, por:

$$BIC=-n log(SQ_{Res}/n)+log(n)p$$

\begin{itemize}
\item O \textit{BIC} penaliza mais fortemente a complexidade do modelo que o \textit{AIC}
ao substituir \textit{p} por $log(n)$ como fator de penalização.
\item Devemos selecionar modelos com menores valores de \textit{AIC} (ou \textit{BIC}).
\end{itemize}

\noindent Esse dois critérios são os mais conhecidos, mas existem outros como: \textit{Deviance Information Criterion} \cite{spiegelhalter2002bayesian}, \textit{Focused Information Criterion} \cite{claeskens2003focused}, \textit{Watanabe–Akaike information criterion} \cite{watanabe2013widely} e o \textit{Hannan–Quinn information criterion} \cite{hannan1979determination}.






\chapter{Inferência dos parâmetros}
\label{ch:inferencia_dos_parametros}

\noindent Nesse capítulo vamos realizar a inferência dos parâmetros. Como saber a distribuição dos parâmetros, como criar intervalos de confiança para os parâmetros e quais os principais testes de hipóteses.

\section{Distribuição}

\noindent Considere a \autoref{eq:somaerro}, onde para encontrar os parâmetros $\beta_0$ e $\beta_1$ é preciso minimizar os erros, logo os parâmetros estão em função dos erros $\varepsilon_i$, por isso a distribuição de probabilidade de $\beta_0$ e $\beta_1$ dependerá da hipótese adotada sobre a distribuição de probabilidade de $\varepsilon_i$. Em princípio o método de mínimos quadrados ordinários não faz qualquer tipo de suposição sobre a natureza de probabilidade dos erros $\varepsilon_i$, por esse motivo o mais comum é suposição de normalidade. Considerando a hipótese de normalidade da explicitada na \autoref{eq:erro} ($\varepsilon_i$ é normal e identicamente distribuído) os estimadores de mínimos quadrados ordinários seguem uma distribuição normal e são identicamentes distribuídos.

\noindent Dado que $\hat{\beta}_0$ e $\hat{\beta}_1$ estão em função de $\varepsilon_i$, então eles seguem uma distribuição normal com

\begin{itemize}
\centering
    \item $ \text{Média de } \hat{\beta}_0 = \beta_0 \text{ e } \text{Variância de } \hat{\beta}_0 = \sigma_{\hat{\beta}_0}^2 \therefore \boxed{\hat{\beta}_0 \sim N(\beta_0, \sigma_{\hat{\beta}_0}^2)}$
    
    \item $ \text{Média de } \hat{\beta}_1 = \beta_1 \text{ e } \text{Variância de } \hat{\beta}_1 = \sigma_{\hat{\beta}_1}^2 \therefore \boxed{\hat{\beta}_1 \sim N(\beta_1, \sigma_{\hat{\beta}_1}^2)}$
\end{itemize}

\noindent Pelas propriedades da distribuição normal, a variável $W$ que é definida como $W = \dfrac{\hat{\beta}_0 - \beta_0}{\sigma_{\hat{\beta}_0}}$, segue uam distribuição normal padrão com média zero e variância igual a um ($=1$), ou $W \sim N(0,1)$. O mesmo raciocínio se aplica a $\hat{\beta}_1$, com $W = \dfrac{\hat{\beta}_1 - \beta_1}{\sigma_{\hat{\beta}_1}}$ e $W \sim N(0,1)$. Sendo $\sigma_{\hat{\beta}_0}$ e $\sigma_{\hat{\beta}_1}$ os erros padrões de, respectivamente, $\hat{\beta}_0$ e $\hat{\beta}_1$.\\

\noindent \textbf{Em resumo} : Assumindo normalidade dos resíduos, então a distribuição dos parâmetros estimados do modelo de regressão linear simples será um distribuição normal.

\section{Intervalo de confiança}

\subsection{Para os estimadores}

\noindent Como visto anteriormente $W$ é uma variável normal padronizada. Podemos empregar a distribuição normal para afirmações probabilísticas sobre $\beta_1$ contando que a verdadeira variância da população, $\sigma^2$, seja conhecida. Se $\sigma^2$ for conhecida, uma propriedade importante de uma variável normalmente distribuída com média $\mu$  e variância $\sigma^2$ é que a área sob a curva normal entre $\mu \pm \sigma$ corresponde a cerca de 68\%, aquela entre os limites $\mu \pm 2\sigma$ é de cerca de 95\% e a que está entre $\mu \pm 3\sigma$ é de cerca de 99.7\%. Mas raramente é conhecida e, na prática, é determinada pelo estimador não viesado $\hat{\sigma}^2$. Se substituirmos $\sigma$ por $\hat{\sigma}$ podemos escrever a equação
$W = \dfrac{\hat{\beta_1} - \beta_1}{\sigma_{\hat{\beta}_1}}$ como 

\begin{equation}
\label{eq:vart}
    t = \dfrac{\hat{\beta_1} - \beta_1}{\text{ep}(\hat{\beta}_1)} = \dfrac{\text{Estimador - Parâmetro}}{\text{Erro padrão estimado do estimador}}
\end{equation}

\noindent onde $\text{ep}(\hat{\beta}_1) = \hat{\sigma}_{\hat{\beta}_1}$.\\

\noindent Entretanto a variável $t$ não possui uma distribuição normal, mas uma distribuição $t$ com $n-2$ graus de liberdade. Portanto, ao invés de usarmos a distribuição normal, vamos usar a distribuição \textit{t} de \textit{student} para estabelecer um intervalo de confiança para $\beta_1$ como a seguir:

\begin{equation}
\label{eq:interv}
    Pr(-t_{\alpha/2} \leq t \leq t_{\alpha/2}) = 1 - \alpha
\end{equation}

\noindent em que o valor \textit{t} entre as duas desigualdades é o valor \textit{t} dado pela equação \autoref{eq:vart} e $t_{\alpha/2}$ é o valor da variável \textit{t} obtido da distribuição \textit{t} de \textit{student} para um nível de significância $\alpha/2$ e $n-2$ graus de liberdade; muitas vezes é chamado de valor crítico de \textit{t} em um nível de significância de $\alpha/2$. Substituindo \autoref{eq:vart} na \autoref{eq:interv}, obtemos

\begin{equation}
\label{eq:interv2}
    Pr\bigg[-t_{\alpha/2} \leq \dfrac{\hat{\beta_1} - \beta_1}{\text{ep}(\hat{\beta}_1)} \leq t_{\alpha/2}\bigg] = 1 - \alpha
\end{equation}

\noindent Reorganizando a equação \autoref{eq:interv2}, temos

\begin{equation}
\label{eq:interv3}
    Pr[\hat{\beta}_1 - t_{\alpha/2}\text{ep}(\hat{\beta}_1) \leq \beta_1 \leq \hat{\beta}_1 + t_{\alpha/2}\text{ep}(\hat{\beta}_1)] = 1-\alpha
\end{equation}

\noindent A \autoref{eq:interv3} oferece um intervalo de confiança de $100(1-\alpha)\%$ para $\beta_1$ e que pode ser reescrito de forma simplificada como

\begin{equation}
\label{eq:interv4}
    \hat{\beta}_1 \pm t_{\alpha/2}\text{ep}(\hat{\beta}_1)
\end{equation}

\noindent Por analogia, pode-se obter o intervalo de confiança para $\beta_0$, conforme abaixo

\begin{equation}
\label{eq:interv5}
    \hat{\beta}_0 \pm t_{\alpha/2}\text{ep}(\hat{\beta}_0)
\end{equation}

\noindent A característica importante dos intervalos de confiança dados na \autoref{eq:interv4} e na \autoref{eq:interv5}: nos dois casos a \textit{amplitide do intervalo de confiança é proporcional ao erro padrão do estimador}. Quanto maior o erro padrão, maior a amplitude do intervalo de confiança. Em outras palavras, quanto maior o erro do estimador, maior a incerteza da estimação do verdadeiro valor do parâmetro desconhecido. O erro padrão é descrito muitas vezes como uma medida de precisão do estimador (da exatidão com que o estimador mede o verdadeiro valor da população).

\subsection{Para a variância}

\noindent A \autoref{eq:eq49}, sob a hipótese de normalidade, segue uma distribuição $\chi^2$ com $(n-2)$ graus de liberdade, onde

\begin{equation}
    \label{eq:eq61}
    \chi^2 = (n-2)\dfrac{\hat{\sigma}^2}{\sigma^2}
\end{equation}


\noindent portanto podemos usar a distribuição $\chi^2$ para estabelecer um intervalo de confiança para $\sigma^2$:

\begin{equation}
    \label{eq:eq62}
    Pr(\chi^2_{1-\alpha/2} \leq \chi^2 \leq \chi^2_{\alpha/2}) = 1 - \alpha
\end{equation}

\noindent onde o valor da distribuição $chi^2$ na desigualdade é dado pela \autoref{eq:eq61} e os valores críticos $\chi^2_{1-\alpha/2}$ e $\chi^2_{\alpha/2}$ são obtidos da tabela de qui-quadrado para $n-$ graus de liberdade.

\noindent Substituindo \autoref{eq:eq61} na \autoref{eq:eq62} e reorganizando, temos

\begin{equation}
    \label{eq:eq63}
    Pr\bigg[(n-2)\dfrac{\hat{\sigma}^2}{\chi^2_{\alpha/2}} \leq \sigma^2 \leq   (n-2)\dfrac{\hat{\sigma}^2}{\chi^2_{1-\alpha/2}} \bigg]
\end{equation}

o dá o intervalo de confiança $100(1-\alpha)\%$. Na \autoref{fig:chiquadrado} vemos as áreas caudais da distribuição qui-quadrado.


\begin{figure}[H]
\centering
\caption{Áreas caudais da distribuição $\chi^2$}
\includegraphics[scale=.8]{imagens/imagens_cap6/chiquadrado.PNG}
\label{fig:chiquadrado}
\end{figure}

\section{Teste de hipóteses}

\noindent Uma hipótese é uma declaração sobre um parâmetro da população. As duas hipóteses complementares em um problema envolvendo um teste de hipóteses
são chamadas \textit{hipótese nula} e \textit{hipótese alternativa}, denotadas por $H_0$ e $H_1$, respectivamente.\\

Dado um parâmetro populacional $\theta$, o formato geral da hipótese nula e da hipótese alternativa é $H_0$ : $\theta \in \Theta$ e $H_1$ $\theta \in \Theta^{C}$, onde $\Theta$ é um algum subconjunto
do espaço de parâmetros e $\Theta^{C}$ é seu complemento. Por exemplo, suponha que a hipótese nula seja que o verdadeiro valor de $\theta$ é $\theta_0$. Assim,

$$ H_0 : \theta = \theta_0$$

\noindent A hipótese alternativa, considerada aceitável caso $H_0$ seja rejeitada, pode pode ter formas como

$$ H_0 : \theta \neq \theta_0$$
$$ H_0 : \theta > \theta_0$$
$$ H_0 : \theta < \theta_0$$

\noindent a depender das informações do problema. 

\noindent Um procedimento para testar uma hipótese, ou um teste de hipótese, é uma regra que especifica: (a) para quais valores amostrais a decisão aceita $H_0$ como verdadeira; e (b) para quais valores amostrais $H_0$ é rejeitada e $H_1$ é aceita como verdadeira. O subconjunto do espaço amostral para o qual $H_0$ será rejeitada é chamado de \textit{região de rejeição}, ou \textit{região crítica}. O complemento da \textit{região de rejeição} é chamado de \textit{região de aceitação}.\\

\noindent Se a hipótese alternativa 
\begin{itemize}
    \item testar desigualdade, o teste será bicaudal (a distribuição terá dois valores críticos);
    
    \item testar se o parâmetro é maior que um valor, o teste será unicaudal na direita (um valor crítico à direita);
    
    \item estar se o parâmetro é menor que um valor, o teste será unicaudal na esquerda (um valor crítico à esquerda).
\end{itemize}

\subsection{\textit{p}-valor}

Depois que um teste de hipóteses é realizado, as conclusões devem ser relatadas de algum modo estatisticamente significativo. Um método para relatar os resultados de um teste é expor o nível de significância $\alpha$ utilizado e a decisão
de rejeitar ou aceitar $H_0$. Se $\alpha$ for pequeno, a decisão de rejeitar $H_0$ é bastante convincente, mas se $\alpha$ for grande, a decisão de rejeitar $H_0$ não é muito convincente porque o teste tem uma grande probabilidade de levar, incorretamente, a esta decisão.

Outro meio de relatar os resultados de um teste é expor o chamado \textit{p}-valor do teste. O \textit{p}-valor $p(X)$ é uma estatística que satisfaz $0 < p(x) < 1$ para cada ponto amostral $x$, e corresponde à probabilidade de ocorrer valores da estatística de teste $W(X)$ mais extremos do que o observado para x, sob a hipótese de $H_0$ ser verdadeira. Ou seja,

$$ p(x) = P(W(X) \leq W(x)| \theta \in \Theta)  $$

\noindent Rejeitaremos $H_0$ para aqueles níveis de significância $\alpha$ maiores do que o p-valor
encontrado.

\section{Testes estatísticos para o modelo de regressão linear simples}

\subsection{Teste para significância dos parâmetros}
\label{ch:teste1}
\noindent Esse teste serve para saber se os parâmetros do modelo de regressão linear simples são estatisticamente diferentes de zero. Como observado na \autoref{eq:vart} 

$$t = \dfrac{\hat{\beta}_1 - \beta_1}{\text{ep}(\hat{\beta}_1)}$$

\noindent \textit{t} segue uma distribuição \textit{t} de \textit{student} com $n-2$ graus de liberdade. As nossas hipóteses (nulas e alternativa) para $\beta_1$ (também vale para $\beta_0$) é que ele é estatisticamente igual a zero.

$$H_0: \beta_1 = 0 \qquad H_1: \beta_1 \neq 0$$

\noindent Observação: nesse caso o nosso teste será bicaudal, pois $H_1$ não é uma igualdade.\\

\noindent Sob $H_0$ ($\beta_1 = 0$) temos que

$$t = \dfrac{\hat{\beta}_1 - 0}{\text{ep}(\hat{\beta}_1)} =  \dfrac{\hat{\beta}_1 }{\text{ep}(\hat{\beta}_1)}$$

\noindent \textbf{Regra de rejeição}: Rejeita-se $H_0$, para um nível de significância $\alpha$, se $|t| > t_{\alpha/2, (n-2)}$. Caso queira verificar a tabela \textit{t} de \textit{student} basta clicar \href{https://sites.icmc.usp.br/francisco/SME0123/listas/Tabela_Dist_t.pdf}{aqui}.

\subsection{Teste de significância da variância}

\noindent Testar a significância estatística do valor estimado da variância dos erros $\hat{\sigma}^2$. Queremos testar a hipótese nula $H_0: \hat{\sigma}^2 = \sigma^2$ contra a hipótese alternativa $H_1: \hat{\sigma}^2 \neq \sigma^2$. Conforme a \autoref{eq:eq61}

\begin{equation}
    \chi^2 = (n-2)\dfrac{\hat{\sigma}^2}{\sigma^2}
\end{equation}

\noindent Se, para um determinado nível de significância $\alpha$, $\chi^2 > \chi^2_{\alpha/2;(n-2)}$ ou  $\chi^2 < \chi^2_{(1-\alpha)/2;(n-2)}$, então rejeitamos a hipótese nula e aceitamos a hipótese alternativa.

\subsection{Teste para significância $SQ_{Reg}$ - Análise de Variância (\textit{ANOVA})}

\noindent Conforme a \autoref{eq:eq53} no \autoref{ch:avaliacao_do_modelo}, a soma dos quadrados totais $SQ_{Totais}$ é decomposta em outros dois componentes: soma dos quadrados explicados ou da regressão $SQ_{Reg}$ e a soma dos quadrados dos resíduos $SQ_{Res}$. Um estudo desses elementos da $SQ_{Totais}$ é conhecido como \textbf{análise de variância} (\textit{ANOVA}) do ponto de vista da regressão.

\noindent Associados a esses componentes estão seus graus de liberdade, número de observações menos o número de parâmetros. A $SQ_{Totais}$ tem $n-1$ graus de liberdade, porque perde-se $1$ grau de liberdade ao calcular $\overline{y}$. A $SQ_{Res}$ tem $n-2$ graus de liberdade (para o modelo de regressão simples, que possui dois parâmetros - $\beta_0$ e $\beta_1$). A $SQ_{Reg}$ tem $1$ grau de liberdade. 

\begin{table}[H]
\centering
\caption{Componentes e seus graus de liberdade}
\begin{tabular}{lllll}
\hline
 $SQ_{Totais}$ & =  & $SQ_{Reg}$  & +   &  $SQ_{Res}$\\
 $(n-1)$ & =   & $1$ & + & $(n-2)$  \\ \hline
\end{tabular}%
\label{Tab: tabela4}
\end{table}

\noindent \textbf{Nota}: $SQ_{Reg} = \beta_1^2 \sum_{i=1}^{n} (x_i - \overline{x})^2$. Logo se $\beta_1$ for estatisticamente não significativo, então $SQ_{Reg}$ será estatisticamente não significativo. Isso implica que $x$ não tem nenhum influência linear sobre $y$.\\

\noindent Para testar a significância de $SQ_{Reg}$ usamos

\begin{equation}
    \label{eq:eq64}
    F = \dfrac{SQ_{Reg}/(p-1)}{SQ_{Res}/(n-p)}
\end{equation}

\noindent Como o modelo de regressão linear simples tem dois parâmetros

\begin{equation}
    \label{eq:eq65}
    F = \dfrac{SQ_{Reg}/(2-1)}{SQ_{Res}/(n-2)} = \dfrac{SQ_{Reg}}{SQ_{Res}/(n-2)}
\end{equation}

\begin{equation}
    \label{eq:eq66}
    F = \dfrac{SQ_{Reg}}{SQ_{Res}/(n-2)} \sim F_{1,n-2}
\end{equation}

\noindent \textbf{Regra de rejeição}: Rejeita-se $H_0: \beta_1 = \beta_2 = \dots = \beta_n = 0$ ($H_1 : \beta_n \neq 0$) para um nível de significância $\alpha$ se $F > F_{\alpha;1,(n-2)}$.

\newpage

\noindent Na \autoref{Tab:tabela5} abaixo temos uma típica análise de variância (\textit{ANOVA}).

\begin{table}[H]
\caption{ANOVA}
\centering
\begin{tabular}{cccccc}
\hline
\begin{tabular}[c]{@{}c@{}}\textbf{Partição} \\ \textbf{da variância}\end{tabular} & \textbf{SQ} & \textbf{gl} & \textbf{Média SQ} & \textbf{F} \\ \hline
\\
$SQ_{Reg}$ & $\sum_{i=1}^{n} (\hat{y}_i)^2 = \hat{\beta}_1^2 \sum_{i=1}^{n} (x_i)^2 $ & $1$ & $\hat{\beta}_1^2 \sum_{i=1}^{n} (x_i)^2$ & $F = \dfrac{SQ_{Reg}}{SQ_{Res}/(n-2)}$\\
\\
$SQ_{Res}$ & $\sum_{i=1}^{n} \hat{\varepsilon}_i^2$ & $(n-2)$ & $\dfrac{\sum_{i=1}^{n} \varepsilon_i^2}{n-2} = \hat{\sigma}^2$ &  \\
\\
$SQ_{Totais}$ & $\sum_{i=1}^{n} y_i^2$ & $(n-1)$ &  & \\ \hline
\end{tabular}%
\label{Tab:tabela5}
\end{table}

\subsection{Teste para hipótese de normalidade}
\label{ch:testnorm}

\noindent Como citado no \autoref{ch:pressupostos}, existem diversos testes estatísticos para testar a hipótese de normalidade dos resíduos, um dos mais conhecidos é o Teste de Normalidade \textit{Jarque-Bera} \cite{jarque1987test}. O teste \textit{Jarque-Bera} testa se a distribuição dos dados segue uma distribuição normal ($H_{0}$) em comparação com uma hipótese alternativa ($H_{1}$) em que os dados seguem alguma outra distribuição. A estatística do teste é baseada em dois momentos dos dados, a assimetria e a curtose, e possui uma $\chi^{2}_{2;1-\alpha}$ distribuição assintótica.\\

\noindent A estatística do teste \textit{Jarque-Bera} é dada pela equação abaixo:

\begin{equation}
    \label{eq:eq67}
    S_{JB} = T\bigg[\frac{\alpha_{1}^{2}}{6} + \frac{\big(\alpha_{2} - 3\big)^{2}}{24} \bigg]
\end{equation}

\noindent onde $\alpha_{1}$ é a medida padrão de assimetria de uma distribuição ou o \textbf{coeficiente de assimetria} e $\alpha_{2}$ o coeficiente de curtose. Ambos podem ser estimados por

$$ \hat{\alpha_1} = \dfrac{\sum_{i=1}^n \varepsilon_i^3}{n(\hat{\sigma}^2)^{3/2}} = \dfrac{\hat{\mu}^3}{n(\hat{\sigma}^2)^{3/2}} \qquad \hat{\alpha_2} = \dfrac{\sum_{i=1}^n \varepsilon_i^4}{n(\hat{\sigma}^2)^2} = \dfrac{\hat{\mu}^3}{n(\hat{\sigma}^2)^2}
$$

\noindent $\dfrac{\sum_{i=1}^n \varepsilon_i^3}{n}$ e $\dfrac{\sum_{i=1}^n \varepsilon_i^4}{n}$ são os estimadores consistentes para, respectivamente, os terceiro e quarto momentos amostrais. Assim podemos reescrever a \autoref{eq:eq67} como

\begin{equation}
    \label{eq:eq68}
    S_{JB} = T\bigg[\frac{\hat{\alpha}_{1}^{2}}{6} + \frac{\big(\hat{\alpha}_{2} - 3\big)^{2}}{24} \bigg]
\end{equation}

\noindent \textbf{Regra de rejeição} : Se $S_{JB} > \chi^{2}_{2;1-\alpha}$, devemos rejeitar $H_{0}$. Uma vantagem desse teste é que ele pode ser desmembrado em outros dois testes: Um para assimetria e outro para curtose com distribuição $\chi^{2}_{1;1-\alpha}$ para cada um.\\

O teste \textit{Jarque-Bera} é bastante conhecido, porém outros também podem ser aplicados para testar normalidade como o \textit{Shapiro-Wilk} \cite{shapiro1965analysis} e o \textit{Anderson-Darling} \cite{anderson1954test}, que possuem a mesma hipótese nula do \textit{Jarque-Bera} e utilizados nesse trabalho

\subsection{Teste para autocorrelação}
\label{ch:testedw}
\noindent Para detectar autocorrelação ou a hipótese de dependência entre os resíduos, o teste mais comumente usado é o \textit{Durbin-Watson} \cite{durbin1992testing}. As hipóteses do teste são $H_0: \rho = 0$ (não há autocorrelação nos resíduos) contra a $H_1: \rho \neq 0$ (há autocorrelação residual).\\

\noindent A estatística do teste é dada por

\begin{equation}
    \label{eq:eq69}
    d = \dfrac{\sum_{i=2}^n (\hat{\varepsilon}_i - \hat{\varepsilon}_{i-1})^2}{\sum_{i=2}^n \hat{\varepsilon}_i^2}
\end{equation}

\noindent a estatística $d$ varia entre 0 e 4 e interpretamos o seus resultados como:

\begin{itemize}
    \item se o valor do teste estiver próximo de 4 ($d \approx 4$), então há evidência para autocorrelação negativa;
    
    \item se estiver próximo de 2 ($d \approx 2$), evidência para ausência de autocorrelação;
    
    \item e se estiver perto de 0 ($d \approx 0$), há evidência para autocorrelação positiva.
\end{itemize}

\noindent Existem outros testes para identificação de autocorrelação como, o teste de \textit{Wallis} \cite{wallis1972testing}, teste de \textit{Breush-Godfrey} \cite{godfrey1978testing}, \textit{Ljung-Box} \cite{ljung1978measure}, etc.


\subsection{Teste para identificar heterocedasticidade}

\noindent Principais formas de identificar heterocedasticidade:

\begin{itemize}
    \item \textbf{Forma gráfica}: gerando uma gráfico de dispersão com os resíduos elevados ao quadrado e a variável explicativa $x$.
    
    \item \textbf{Testes estatísticos}: Testes \textit{Goldfeld-Quandt}, \textit{Breusch-Pagan} ou de \textit{White}.
\end{itemize}

\noindent Nesse trabalho usaremos os testes \textit{Breusch-Pagan} e \textit{Goldfeld-Quandt}, cuja a hipótese nula é que os resíduos são homocedásticos (variância constante). Se o $p$-valor do teste for maior que o nível de significância (por padrão é 5\%), então aceitamos a hipótese nula, caso contrário os resíduos são heterocedásticos.

\subsubsection{Teste \textit{Breusch-Pagan}}

\noindent Falando um pouco sobre o teste \textit{Breusch-Pagan} a ideia desse teste é gerar uma regressão liner entre os resíduos quadrados ($\varepsilon^2$) do modelo e a variável explicativa ($x$), para testar se os resíduos tem relação com o regressor. Temos então

$$ \hat{\varepsilon}^2_i = \delta_0 + \delta_1 x_i + u_i$$

\noindent onde $u_i$ são os resíduos desse modelo. É verificado se $\delta_1$ é estatisticamente significativo (se $\delta_1 \neq 0$). Se $\delta_1 \neq 0$, então a variância dos resíduos dependem da variável explicativa; mas se $\delta_1 = 0$, então teremos do modelo apenas $\delta_0$ e $u_i$ ($E(u_i) = 0$), então a variância é contante, pois $\delta_0$ é constante.














\chapter{Exemplos}
\label{ch:exemplos}

\noindent Nesse capítulo serão realizados as etapas dos capítulos anteriores para gerar um modelo de regressão linear simples na linguagem \textit{R}. Iremos seguir todas as etapas dos capítulos anteriores, desde a estimação do modelo até a realização dos testes estatísticos para avaliar o modelo.

\section{Base de dados}

\noindent A base de dados que usaremos será a \textit{mtcars}. Os dados foram extraídos da revista \textit{Motor Trend US} de 1974 e abrangem o consumo de combustível e 10 aspectos do \textit{design} e desempenho de automóveis para 32 automóveis (modelos de 1973-74).\\

\noindent Na \autoref{Tab: tabela6} temos o nome das variáveis na base de dados e as suas respectivas descrições.

\begin{table}[H]
\caption{Variáveis do \textit{dataset mtcars}}
\centering
\begin{tabular}{cl}
\textbf{Nome da coluna} & \textbf{Descrição da variável} \\ \hline
\textit{mpg} & Milhas/galão (EUA)\\
\textit{cyl} & Número de cilindros\\
\textit{disp} & Deslocamento (\textit{cu.in.}) \\
\textit{hp} & Potência bruta \\
\textit{drat} & Relação do eixo traseiro \\
\textit{wt} & Peso (1000 libras) \\
\textit{qsec} & Tempo de $1/4$ de milha \\
\textit{vs} &  Motor (0 = em forma de V, 1 = reto) \\
\textit{am} & Transmissão (0 = automática, 1 = manual)\\
\textit{gear} & Número de marchas para frente\\ \hline
\end{tabular}
\label{Tab: tabela6}
\end{table}

\noindent Para gerar o modelo vamos usar como variável dependente ou alvo a coluna \textit{mpg} (milha por galão), que é uma medida de consumo de combustível do veículo, e como variável explicativa usaremos a variável \textit{hp} (\textit{horse power} ou cavalos de força), uma medida de potência do veículo.\\

\noindent O objetivo é analisar a variação média do consumo do veículo a medida que a potência do mesmo varia.
\newpage

\section{Comandos \textit{R}}

\noindent Realizando a instalação dos pacotes que serão usados nesse capítulo. No pacote \textit{lmtest} encontram-se os teste para avaliarmos o modelo. No pacote \textit{tseries} o teste \textit{Jarque-Bera} para analisar a hipótese de normalidade dos resíduos. E o pacote \textit{ggplot2} para gerar gráficos, se preciso. No \autoref{lst:code1} realizamos a instalação dos pacotes.


\begin{lstlisting}[language=R, caption = {Instalação dos pacotes}, label={lst:code1}]
#instalando os pacotes
library(lmtest)
library(tseries)
library(nortest)
library(olsrr)
\end{lstlisting}

\noindent Primeiras linhas da base de dados

\begin{lstlisting}[language=R, caption = {Modelo de regressão},label={lst:code2}]
#primeiras linhas da base de dados
head(mtcars)
\end{lstlisting}

\begin{figure}[H]
\centering
\caption{Primeiras linhas da base de dados}
\includegraphics[scale=1]{imagens/imagens_cap7/head.PNG}
\label{fig:modelo2}
\end{figure}

\noindent Últimas linhas da base de dados

\begin{lstlisting}[language=R, caption = {Modelo de regressão},label={lst:code3}]
#linhas finais da base de dados
tail(mtcars)
\end{lstlisting}

\begin{figure}[H]
\centering
\caption{Últimas linhas da base de dados}
\includegraphics[scale=1]{imagens/imagens_cap7/tail.PNG}
\label{fig:modelo3}
\end{figure}

\newpage

\noindent No \autoref{lst:code6} temos o sumário das colunas que selecionamos.

\begin{lstlisting}[language=R, caption = {Sumário das variáveis},label={lst:code6}]
#linhas finais da base de dados
summary(data.frame(mtcars$mpg,mtcars$hp))
\end{lstlisting}

\begin{figure}[H]
\centering
\caption{Sumário das colunas}
\includegraphics[scale=1]{imagens/imagens_cap7/sumario1.PNG}
\label{fig:modelo4}
\end{figure}

\noindent No \autoref{lst:code4} vemos a correlação entre as variáveis \textit{mpg} e \textit{hp}.

\begin{lstlisting}[language=R, caption = {Tabela de correlação},label={lst:code4}]
cor(data.frame(mtcars$mpg,mtcars$hp))
\end{lstlisting}

\begin{figure}[H]
\centering
\caption{Correlação}
\includegraphics[scale=1]{imagens/imagens_cap7/correlacao.PNG}
\label{fig:modelo4}
\end{figure}

\noindent No \autoref{lst:code5} temos o teste para saber se a correlação entre as variáveis é estatisticamente significativa.

\begin{lstlisting}[language=R, caption = {Sumário das variáveis},label={lst:code5}]
#linhas finais da base de dados
summary(data.frame(mtcars$mpg,mtcars$hp))
\end{lstlisting}

\begin{figure}[H]
\centering
\caption{Teste de correlação}
\includegraphics[scale=1]{imagens/imagens_cap7/corteste.PNG}
\label{fig:modelo4}
\end{figure}

\noindent Para estimar um modelo de regressão linear simples no \textit{R} basta usar o \autoref{lst:code5}.

\begin{lstlisting}[language=R, caption = {Modelo de regressão},label={lst:code7}]
#gerando o modelo
modelo = lm(mpg~hp, data=mtcars)
modelo
\end{lstlisting}

\begin{figure}[H]
\centering
\caption{Modelo de regressão linear simples}
\includegraphics[scale=1]{imagens/imagens_cap7/modelo2.PNG}
\label{fig:modelo5}
\end{figure}

\noindent Temos os coeficientes $\beta_0$ e $\beta_1$ cujo os valores são, respectivamente, $30.09886$ e $-0.06823$. A interpretação do modelo é: \textbf{Variação de uma unidade de cavalo de potência do veículo diminui, em média, $0.06823$ unidade o consumo de combustível do veículo, em Milhas por galão.}\\

\noindent Após gerar o modelo podemos gerar o sumário com as informações do mesmo. O comando \textit{summary( )} no \autoref{lst:code3} gera a saída na \autoref{fig:modelo6}.

\begin{lstlisting}[language=R, caption = {Sumário do modelo},label={lst:code8}]
#resumo do modelo
summary(modelo)
\end{lstlisting}

\begin{figure}[H]
\centering
\caption{Sumário do modelo}
\includegraphics[scale=.93]{imagens/imagens_cap7/sumario.PNG}
\label{fig:modelo6}
\end{figure}

\subsection{Interpretanto o sumário do modelo}

\noindent A primeira informação impressa pelo resumo de regressão linear após a fórmula é a estatística de resumo residual. As estatísticas de resumo residual fornecem informações sobre a simetria da distribuição residual. A mediana deve ser próxima a zero com a média esperada dos resíduos sendo zero, sendo a distribuição simétrica ambos serão iguais. O terceiro quartil ($3Q$) e o primeiro quartil ($1Q$) devem ser próximos um do outro em magnitude, assim como o máximo e o mínimo.


\begin{figure}[H]
\centering
\caption{Estatísticas dos resíduos}
\includegraphics[scale=1]{imagens/imagens_cap7/residuals2.PNG}
\label{fig:modelo7}
\end{figure}

\noindent Na \autoref{fig:modelo7} mostra que a mediana está próxima de zero, mas os quartis e o máximo e o mínimo não estão próximos. Mais a frente serão realizados os teste de normalidade que verificarão a normalidade, entretanto os resíduos não aparentam seguir uma distribuição normal.\\

\noindent Agora temos os coeficientes do modelo que já foram interpretados anteriomente. Nessa parte do sumário podemos ver se os parâmetros são ou não estatisticamente significativos (estatisticamente diferentes de zero). Como vista na \autoref{ch:teste1}, testa-se a hipótese deles serem iguais a zero, se o valor do $p$-valor for menor que o nível de significância $\alpha = 0.05 (5\%)$ (padrão), então rejeitamos a hipótese nula. 

\begin{figure}[H]
\centering
\caption{Coeficientes do modelo}
\includegraphics[scale=1]{imagens/imagens_cap7/coeficientes2.PNG}
\label{fig:modelo8}
\end{figure}

\noindent A \autoref{fig:modelo8} mostra o valor dos coeficientes (\textit{Estimated}), o erro padrão (\textit{Std. Error}), o valor da estatística \textit{t} sob a hipótese nula (\textit{t value} - razão entre o valor do parâmetro e o erro padrão) e os $p$-valores ($\text{Pr}(>|t|)$). Todos os $p$-valores ficaram abaixo do nível de significância de $5\%$ ), então rejeitamos $H_0$ e os nossos coeficientes são estatisticamente significativos e diferentes de zero.

\newpage

\noindent \textbf{Erro padrão residual} : fornece o desvio padrão dos resíduos e nos informa sobre quão grande é o erro de predição na amostra ou nos dados de treinamento . Na \autoref{fig:modelo9} temos o valor dessa medida : $3.863$.

\begin{figure}[H]
\centering
\caption{Erro padrão residual}
\includegraphics[scale=1]{imagens/imagens_cap7/epr.PNG}
\label{fig:modelo9}
\end{figure}

\noindent \textbf{Coeficiente de determinação - $R^2$ e $R^2_{\text{ajustado}}$}: Como já explicado na \autoref{ch:r2}  e na \autoref{ch:r2a}, o $R^2$ mede o quão bem ajustado está o nosso modelo e o $R^2_{\text{ajustado}}$ mede apenas a variação das variáveis que de fato são relevantes para o modelo. A \autoref{fig:modelo10} vemos que o valor do $R^2$ foi de $0.6024$ e do $R^2_{\text{ajustado}}$ foi um pouco menor, $0.5892$.

\begin{figure}[H]
\centering
\caption{Coeficiente de determinação}
\includegraphics[scale=1]{imagens/imagens_cap7/r22.PNG}
\label{fig:modelo10}
\end{figure}

\noindent \textbf{Estatística $F$}: A estatística $F$ gerada pelo sumário é para testar se pelo menos um dos coeficientes é estatisticamente igual a zero. O $p$-valor na \autoref{fig:modelo11} ficou abaixo de $5\%$, logo rejeita-se $H_0$ de que todos os parâmetros são conjuntamente estatisticamente não significativos. (\textbf{Observação}: como estamos usando uma regressão linear simples era de se esperar o resultado encontrado, mas essa estatística é melhor aplicada quando utiliza-se uma regressão linear múltipla).

\begin{figure}[H]
\centering
\caption{Estatística \textit{F}}
\includegraphics[scale=1]{imagens/imagens_cap7/Fstats2.PNG}
\label{fig:modelo11}
\end{figure}

\subsection{Análise da variância - \textit{ANOVA}}

\noindent Tabela \textit{ANOVA} do modelo com a soma dos quadrados explicados da regressão ($SQ_{Reg}$).

\begin{lstlisting}[language=R, caption = {ANOVA},label={lst:code9}]
anova(modelo)
\end{lstlisting}

\begin{figure}[H]
\centering
\caption{Análise de variância - ANOVA}
\includegraphics[scale=.93]{imagens/imagens_cap7/anova.PNG}
\label{fig:modelo12}
\end{figure}

\newpage

\noindent Na \autoref{fig:modelo13} temos o gráfico de consumo \textit{vs} potência com a reta de regressão do nosso modelo.

\begin{lstlisting}[language=R, caption = {Consumo \textit{vs} Potência},label={lst:code10}]
plot(mtcars$mpg~mtcars$hp, main='Milhas/Galao_vs_Potencia',
xlab = "Potencia", ylab = "Milhas/Galao")
abline(lm(mpg~hp, data=mtcars), col="red")
\end{lstlisting}

\begin{figure}[H]
\label{fig:consumovspotencia}
\caption{Gráfico - Consumo \textit{vs} Potência}
\includegraphics[scale=.9]{imagens/imagens_cap7/consumovspotencia.PNG}
\label{fig:modelo13}
\end{figure}

\noindent A linha reta vermelha do gráfico é dada pela função

\begin{equation}
\label{eq:eq71}
    \text{Consumo (Milha/galão) = 30.09886 - 0.06823 \text{Potência (Cavalos de força)}}
\end{equation}

\newpage

\noindent Informações do modelo podem ser extraídas, conforme o \autoref{lst:code11}.

\begin{lstlisting}[language=R, caption = {Informações do modelo},label={lst:code11}]
#coeficientes
modelo$coefficients
#residuos
modelo$residuals
#valores treinados do modelo
modelo$fitted.values
#graus de liberdade dos residuos
modelo$df.residual
#formula do modelo no R
modelo$call
\end{lstlisting}

\noindent Os intervalos de confiança dos parâmetros, para diferentes valores de $\alpha$ (10\%, 5\% e 1\%), podem ser obtidos conforme códigos abaixo.

\begin{lstlisting}[language=R, caption = {IC com $\alpha$ a 10\%},label={lst:code12}]
confint(modelo, level = 0.90)
\end{lstlisting}

\begin{figure}[H]
\centering
\caption{Intervalo de confiança - $\alpha$ = 10\%}
\includegraphics[scale=1]{imagens/imagens_cap7/ic10.PNG}
\label{fig:ic10}
\end{figure}

\begin{lstlisting}[language=R, caption = {IC com $\alpha$ a 5\%},label={lst:code13}]
confint(modelo)
\end{lstlisting}

\begin{figure}[H]
\centering
\caption{Intervalo de confiança - $\alpha$ = 5\%}
\includegraphics[scale=1]{imagens/imagens_cap7/ic5.PNG}
\label{fig:ic5}
\end{figure}

\begin{lstlisting}[language=R, caption = {IC com $\alpha$ a 1\%},label={lst:code14}]
confint(modelo, level = 0.99)
\end{lstlisting}

\begin{figure}[H]
\centering
\caption{Intervalo de confiança - $\alpha$ = 1\%}
\includegraphics[scale=1]{imagens/imagens_cap7/ic1.PNG}
\label{fig:ic1}
\end{figure}

\noindent Na \autoref{fig:treinadoxreal} gerada pelo \autoref{lst:code15}, vemos o ajuste entre os dados reais e os valores gerados pelo modelo.

\begin{lstlisting}[language=R, caption = {Valores reais vs valores treinados},label={lst:code15}]
plot(mtcars$mpg, modelo$fitted.values, 
main="Valores reais vs Valores treinados",
     xlab = "Valores treinados",ylab = "Valores reais")
abline(lm(mtcars$mpg~modelo$fitted.values),col="red")
\end{lstlisting}

\begin{figure}[H]
\centering
\caption{Valores reais vs Valores treinados}
\includegraphics[scale=.9]{imagens/imagens_cap7/reaisxtreinados.PNG}
\label{fig:treinadoxreal}
\end{figure}


\subsection{Análise dos resíduos}

\noindent Aqui nessa subseção iremos avalisar o modelo por meio dos seus resíduos e aplicar testes para verificar as principais hipóteses e pressupostos do modelo de regressão linear simples.

\noindent Primeiramente iremos

\begin{itemize}
    \item Criar gráficos sobre os resíduos como \textit{boxplot}, \textit{qqplot}, histograma, etc;
    
    \item Realizar testes estatísticos para  autocorrelação, heterocedasticidade e normalidade dos resíduos.
\end{itemize}

\newpage

\noindent O \autoref{lst:code16} gera os três gráficos da \autoref{fig:histboxqqplot} onde há o histograma, o \textbf{boxplot} e o \textit{qqplot} dos resíduos.

\begin{lstlisting}[language=R, caption = {Histograma, \textit{Boxplot}, \textit{QQplot}},label={lst:code16}]
par(mfrow=c(1,3))
hist(residuals(modelo), 
     main="Histograma dos residuos", 
     xlab = "Residuos", ylab = ""); 
boxplot(residuals(modelo), 
        main="Boxplot dos residuos"); 
qqnorm(residuals(modelo), 
       main="QQplot dos residuos", xlab = "Quantis teoricos"); 
qqline(residuals(modelo))
\end{lstlisting}

\begin{figure}[H]
\centering
\caption{Histograma, \textit{Boxplot}, \textit{QQplot} dos resíduos}
\includegraphics[scale=.9]{imagens/imagens_cap7/histboxqqplot.PNG}
\label{fig:histboxqqplot}
\end{figure}

\noindent O histograma mostra a distribuição dos resíduos, que deveria ter um formato de sino, típico de uma distribuição normal; mas não é o que o gráfico mostra. O \textit{boxplor} deveria ter a linha preta (que representa a mediana) centrada no valor zero, mas está deslocado um pouco abaixo e o limite superior do gráfico indica presenta de \textit{outliers} nos resíduos. Por fim, o \textit{qqplot} deveria mostrar todos os quantis teóricos bem próximos da linha reta do gráfico para indicar normalidade dos resíduos, mas há muitos pontos distantes da reta. Esse resultados, a princípio, indicam que os resíduos não seguem uma distribuição normal.

\subsubsection{Testes de normalidade}

\noindent Iremos testar a hipótese de normalidade com os testes mencionados na \autoref{ch:testnorm}. O primeiro, que foi explicado, é o \textit{Jarque-Bera} e o resultado para os resíduos do nosso modelo pode ser vista a seguir.

\begin{lstlisting}[language=R, caption = {Teste \textit{Jarque-Bera}},label={lst:code17}]
jarque.bera.test(residuals(modelo))
\end{lstlisting}

\begin{figure}[H]
\centering
\caption{Resultado do teste \textit{Jarque-Bera}}
\includegraphics[scale=1]{imagens/imagens_cap7/jb.PNG}
\label{fig:jb}
\end{figure}

\noindent O teste \textit{Jarque-Bera} apresentou um $p$-valor maior que 5\% indicando que há normalidade nos resíduos, mas esse teste funciona melhor quando o número de observações é grande, o que não é o nosso caso. Por esse motivo iremos utilizar os testes \textit{Shapiro-Wilk} e o \textit{Anderson-Darling}. O \autoref{lst:code18} e gera os resultado da \autoref{fig:swad}.


\begin{lstlisting}[language=R, caption = {Testes \textit{Shapiro-Wilk} e \textit{Anderson-Darling}},label={lst:code18}]
shapiro.test(residuals(modelo))
ad.test(residuals(modelo))
\end{lstlisting}

\begin{figure}[H]
\centering
\caption{Resultados dos testes \textit{Shapiro-Wilk} e \textit{Anderson-Darling}}
\includegraphics[scale=1]{imagens/imagens_cap7/swad.PNG}
\label{fig:swad}
\end{figure}

\noindent Os $p$-valores desses dois testes ficaram abaixo do nível de significância de 5\% ($0.02568$ e $0.03447$), logo rejeitamos a hipótese nula de normalidade nos resíduos. 

\newpage

\subsubsection{Testes de heterocedasticidade}

\noindent Vamos testar se os resíduos possuem ou não variância normal. O \autoref{lst:code19} gera os resultados da \autoref{fig:testhetero}.\\

\noindent O teste \textit{Goldfeld-Quandt} separa os resíduos em duas partes (por padrão a proporção entre essas duas parte é igual - \textit{fraction} = 1/2), em seguida cria duas regressões entre os resíduos quadrados e variável explicativa e testa se a variância dessas regressões são iguais. Se forem, homocedasticidade, caso contrário, heterocedasticidade.

\begin{lstlisting}[language=R, caption = {Testes \textit{Goldfeld-Quandt} e \textit{Breusch-Pagan}},label={lst:code19}]
#teste Goldfeld-Quandt
gqtest(modelo, fraction=1/2, order.by = ~hp, data=mtcars)
#teste Breusch-Pagan
bptest(modelo, varformula = ~hp, data=mtcars, studentize = F)
\end{lstlisting}

\begin{figure}[H]
\centering
\caption{Resultados dos testes \textit{Goldfeld-Quandt} e \textit{Breusch-Pagan}}
\includegraphics[scale=1]{imagens/imagens_cap7/testehetero.PNG}
\label{fig:testhetero}
\end{figure}

\noindent Para ambos os $p$-valores ficaram bem acima de 5\%, logo aceitamos a hipótese nula de homocedasticidade dos resíduos. Seguindo a ideia do teste \textit{Breusch-Pagan} poderíamos observar a significância estatística do coeficiente da regressão dos resíduos quadrados com a variável explicativa \textit{hp}.

\begin{lstlisting}[language=R, caption = {Testes de heterocedasticidade},label={lst:code20}]
residuos2 <- residuals(modelo)^2 #residuos quadrados
summary(lm(residuos2~mtcars$hp)) #regressao
\end{lstlisting}

\begin{figure}[H]
\centering
\caption{Sumário}
\includegraphics[scale=1]{imagens/imagens_cap7/testesigresx.PNG}
\label{fig:testsigresx}
\end{figure}

\noindent Conforme a \autoref{fig:testsigresx} o coeficiente da variável explicativa é estatisticamente não significativo, logo não há relação entre os resíduos quadrados e o a potência dos veículos (\textit{hp}).

\subsubsection{Teste para autocorrelação}

\noindent Conforme a \autoref{ch:testedw} se a estatística do teste \textit{Durbin-Watson} estiver próxima de $2$, então há evidência para ausência de autocorrelação; caso contrário há evidência para a presença de autocorrelação.

\begin{lstlisting}[language=R, caption = {Teste de autocorrelação},label={lst:code21}]
dwtest(modelo$model)
\end{lstlisting}

\begin{figure}[H]
\centering
\caption{Resutado do teste \textit{Durbin-Watson}}
\includegraphics[scale=1.1]{imagens/imagens_cap7/dwteste.PNG}
\label{fig:testedw}
\end{figure}

\noindent Conforme a imagem \autoref{fig:testedw} a estatística não está próxima do valor $2$, então concluímos que há presença de autocorrelação nos resíduos.

\newpage

\subsection{Avaliação do modelo}

\noindent Agora vamos visualizar alguma métricas que citamos no \autoref{ch:avaliacao_do_modelo}, entretanto vale lembrar que elas são usadas para realizar comparação entre modelos. Aqui só vamos deixar os códigos em \textit{R}, visto que não há outros modelos para comparar com o nosso.

\subsubsection{Critério de informação}

\noindent Critérios de informação de \textit{Akaike} e de \textit{Schwarz}, respectivamente, na \autoref{fig:aicbic}.

\begin{lstlisting}[language=R, caption = {Critério de informação},label={lst:code22}]
AIC(modelo) #Akaike
BIC(modelo) #Bayesian/Schwarz
\end{lstlisting}

\begin{figure}[H]
\centering
\caption{Critérios \textit{Akaike} e de \textit{Schwarz}}
\includegraphics[scale=1.2]{imagens/imagens_cap7/aicbic.PNG}
\label{fig:aicbic}
\end{figure}

\subsubsection{\textit{MSE} - Média do quadrados dos erros}


\begin{lstlisting}[language=R, caption = {MSE},label={lst:code23}]
mean(residuos2) #residuos2 foi calculado anteriormente
\end{lstlisting}

\noindent Temos como resultado o valor de $13.98982$. Vale salientar que existem outras métricas que podem ser calculados usando o pacote \textit{Metrics} (clique \href{https://cran.r-project.org/web/packages/Metrics/Metrics.pdf}{\textbf{aqui}})

\subsubsection{Estatística \textit{PRESS}}

\begin{lstlisting}[language=R, caption = {Estatística \textit{PRESS}},label={lst:code24}]
PRESS(modelo)
\end{lstlisting}

\begin{figure}[H]
\centering
\caption{Resultado da statística \textit{PRESS}}
\includegraphics[scale=0.9]{imagens/imagens_cap7/press.PNG}
\label{fig:press}
\end{figure}











\chapter{Conclusão}
\label{ch:conclusao}

\noindent Esse trabalho faz parte da atividade 5 da disciplina \textbf{Uso de Editores de Texto na Elaboração de Documentos Científicos} do professor Dr. Antonio Samuel, sobre o modelo de \textbf{regressão linear simples}. O trabalho foi criado no editor de textos científicos \textit{Overleaf} e foram usadas a documentação da classe \textit{abntex2} e do pacote \textit{abntex2cite}. Todos os capítulos contemplam os tópicos exigidos na atividade.


\citeoption{abnt-emphasize=bf}
\bibliography{referencias/references}
\end{document}



