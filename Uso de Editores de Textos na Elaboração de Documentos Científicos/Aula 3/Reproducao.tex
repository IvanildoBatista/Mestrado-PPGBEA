\documentclass[a4paper,notitlepage]{book}
\usepackage[lmargin=2cm,tmargin=2cm,rmargin=2cm,bmargin=2cm]{geometry}
\usepackage[utf8]{inputenc}
\usepackage{geometry}
\usepackage[brazilian]{babel}
\usepackage{xcolor}
\usepackage{setspace}
\usepackage[inline]{enumitem}
\usepackage{ulem}
\usepackage{pifont}

\begin{document}

\noindent \textit{\textbf{Exemplo 1.2: Leis de Morgan}}

\hfill Um conjunto de relações entre união e intersecção de conjuntos, conhecidos como Leis de Morgan, auxilia na demonstração de vários resultados.

\noindent Elas são dadas por:

$$ (i) \Big(\bigcup_{i=1}^{n} A_{i} \Big)^c  = \bigcap_{i=1}^{n} A_{i}^c \qquad e \qquad (ii) \Big(\bigcap_{i=1}^{n} A_{i} \Big)^c  = \bigcup_{i=1}^{n} A_{i}^c$$

Vamos verificar a relação \textit{(i)} e deixamos ao leitor a demonstração da outra parte, que é análoga. O caminho usual, para demonstrar igualdades entre conjuntos, é provar que cada um deles está contido no outro. Dessa forma, temos duas partes a serem verificadas:

$$ \Big(\bigcup_{i=1}^{n} A_{i} \Big)^c  \subset \bigcap_{i=1}^{n} A_{i}^c \qquad (parte 1)\qquad e \qquad \Big(\bigcap_{i=1}^{n} A_{i} \Big)^c \supset \bigcup_{i=1}^{n} A_{i}^c \qquad (parte 2).$$

\noindent \textit{Prova da Parte 1:}

Suponha que \textit{w} $\in \Big(\bigcup_{i=1}^{n} A_{i} \Big)^c$. Então \textit{w} $\notin \bigcup_{i=1}^{n} A_{i}$ e, ainda, $w \notin A_i$ para todo

\begin{itemize}
\item[i.] Dessa forma \textit{w} $\in A_{i}^c$ para todo \textit{i} e, consequentemente, \textit{w} $\in \bigcap_{i=1}^{n} A_{i}^c$ 
 
\end{itemize}

\end{document}

